\documentclass[12pt]{extarticle}
\usepackage[export]{adjustbox}
\usepackage{amsmath}
\usepackage{amssymb}
\usepackage{amsthm}
\usepackage{empheq}
\usepackage{fancyhdr}
\usepackage{graphicx}
\usepackage{soul}
\usepackage{subcaption}
\usepackage{subfiles}
\usepackage[most]{tcolorbox}
\usepackage[explicit]{titlesec}
\usepackage{ulem}

\newenvironment{whitebox}{\begin{tcolorbox}[colback=white]}{\end{tcolorbox}}
\newenvironment{defnbox}{\begin{tcolorbox}[colback=blue!3!white]\begin{definition}}{\end{definition}\end{tcolorbox}}
\newenvironment{thmbox}{\begin{tcolorbox}[colback=blue!3!white]\begin{theorem}}{\end{theorem}\end{tcolorbox}}


\newtcbox{\emphbox}[1][]{%
    nobeforeafter, math upper, tcbox raise base,
    enhanced, colframe=black,
    colback=white, boxrule=1pt,
    #1}
\NewDocumentEnvironment{eqnbox}{b}{}{
  \begin{empheq}[box=\emphbox]{equation*}
    #1
  \end{empheq}
}
\NewDocumentEnvironment{alignbox}{b}{}{
  \begin{empheq}[box=\emphbox]{align*}
    #1
  \end{empheq}
}
\NewDocumentEnvironment{gatherbox}{b}{}{
  \begin{empheq}[box=\emphbox]{gather*}
    #1
  \end{empheq}
}


\theoremstyle{definition}
\newtheorem*{corollary}{Corollary}
\newtheorem*{definition}{Definition}
\newtheorem*{lemma}{Lemma}
\newtheorem*{prop}{Prop}
\newtheorem*{recall}{Recall}
\newtheorem*{theorem}{Theorem}
\newtheorem*{thm}{Thm}

\theoremstyle{remark}
\newtheorem*{note}{Note}
\newtheorem*{notation}{Notation}

\newcommand{\pstart}[0]{\noindent}
\newcommand{\newp}[0]{\vspace{8pt}\pstart}
\newcommand{\term}[1]{\textit{\textbf{#1}}}
\newcommand{\ulbf}[1]{\textbf{\ul{#1}}}

\newcommand{\titleul}[1]{\newp\ulbf{#1}\vspace{7pt}~\\}
\newcommand{\ctrrule}[2]{\begin{center}\rule{#1}{#2}\end{center}}


% General Commands
\newcommand{\abs}[1]{\left|#1\right|}
\newcommand{\mbold}[1]{\boldsymbol{#1}}
\newcommand{\image}[1]{\text{Im}\left(#1\right)}
\newcommand{\set}[1]{\left\{#1\right\}}
\newcommand{\simplies}[0]{\;\Rightarrow\;}

\newcommand{\fourier}[1]{\mathcal{F}\left[#1\right]}
\newcommand{\fourierinv}[1]{\mathcal{F}^{-1}\left[#1\right]}

% Fields
\newcommand{\reals}[0]{\mathbb{R}}
\newcommand{\complex}[0]{\mathbb{C}}
\newcommand{\integers}[0]{\mathbb{Z}}
\newcommand{\naturals}[0]{\mathbb{N}}

% Linear Algebra Commands
\newcommand{\inner}[2]{\left<#1,#2\right>}
\newcommand{\norm}[1]{\left|\left|#1\right|\right|}
\newcommand{\rank}[1]{\text{rank}\left(#1\right)}
\newcommand{\trace}[1]{\text{tr}(#1)}
\newcommand{\vect}[1]{\left<#1\right>}
\newcommand{\vspan}[1]{\text{span}\left\{#1\right\}}

% Calculus Commands
\newcommand{\grad}[0]{\nabla}

\graphicspath{ {Notes/Images} }

% \usepackage[lmargin=0.3in,rmargin=0.3in,bmargin=0.3in,tmargin=0.3in]{geometry} % <- Shrunken sizing
\usepackage[lmargin=0.7in,rmargin=0.7in,tmargin=0.7in,bmargin=0.7in]{geometry}
\pagenumbering{gobble}

\titleformat{\section}{\large\bfseries}{\thesection}{1em}{#1\hrule\vspace*{-14pt}}


\begin{document}

% \pstart Stanley Wei

\begin{center}
    \begin{Large}
        \textbf{ECE 102: Systems \& Signals}
    \end{Large}
    
    \begin{large}
        \vspace{8pt}
        Prof. J. Kao $\vert$ Fall 2024
    \end{large}
\end{center}
\tableofcontents


\pagebreak
\section{Signals}
Continuous vs. discrete signals

~\\ \newp
\ulbf{Operations on Signals}: \begin{enumerate}
    \item \term{Amplitude scaling}: $x(t)\mapsto ax(t)$ \begin{enumerate}
        \item[(i)] $0<\abs{a}<1\implies$ ``\textit{attenuation}'', $\abs{a}>1\implies$ ``\textit{amplification}''
        \item[(ii)] $a<0\implies$ ``inversion''
    \end{enumerate}
    \item \term{Time scaling}: $x(t)\mapsto x(at)$ \begin{enumerate}
        \item[(i)] $0<\abs{a}<1\implies$ ``\textit{expansion}'',  $\abs{a}>1\implies$ ``\textit{compression}''
        \item[(ii)] $a<0\implies$ ``\textit{reversal}''
    \end{enumerate}
    \item \term{Time shifting}: $x(t)\mapsto x(t+t_0)$ \begin{enumerate}
        \item[(i)] $x(t-t_0)\implies$ ``\textit{delayed}''; $x(t+t_0)\implies$ ``\textit{advanced}''
    \end{enumerate}
    \item[($\ast$)] Order of operations: reverse PEMDAS
\end{enumerate}

~\\ \newp
\ulbf{Classes of Signals}: \begin{itemize}
    \item \term{Even signals} $x(t)=x(-t)\;\forall\;t$, e.g. $\sin(t)$; \term{odd signals} $x(t)=-x(-t)\;\forall\;t$, e.g. $\cos(t)$ \begin{itemize}
        \item Else, antisymmetric; can be written as sum of even \& odd $x_{e/o}(t)=\frac{1}{2}(x(t)+/-x(-t))$
    \end{itemize}
    \item \term{Periodic signals}: $\exists\;T_0>0$ s.t. $x(t+T_0)=x(t)\;\forall\;t$ \begin{itemize}
        \item Sum of two periodic signals with periods $T_1,T_2$ has period $\text{lcm}(T_1,T_2)$ if both rational; if one rational and one irrational, then not periodic
        \item \textit{Periodic extension}: can take a time-limited/aperiodic signal and repeat it
    \end{itemize}
    \item \term{Causal signals}: nonzero only for $t\geq0$; \textit{anticausal}: nonzero only for $t\leq0$; \textit{noncausal}: nonzero for some $t<0$
\end{itemize}

\newp
\textbf{Sinusoids}: $x(t)=A\cos(\omega t-\theta)$ \begin{itemize}
    \item Variables: angular frequency $\omega=\frac{2\pi}{T}$, amplitude $A$, phase shift $\theta$
    \item Trigonometry rules: \begin{itemize}
        \item $\sin\theta=\cos(\theta-\pi/2)$
        \item $\sin(a+b)=\sin(a)\cos(b)+\cos(a)\sin(b)$
        \item $\cos(a+b)=\cos(a)\cos(b)-\sin(a)-\sin(b)$
        \item $\sin^2(\theta)=(1-\cos(2\theta))/2$
        \item $\cos^2(\theta)=(1+\cos(2\theta))/2$
    \end{itemize}
\end{itemize}

\newp
\textbf{Signal energy \& power}: A signal is either energy, power, or neither \begin{enumerate}
    \item If $0<E<\infty$, $x(t)$ called an \term{energy signal}: \begin{align*}
        E=\lim_{T\to\infty}\int_{-T}^T\abs{x(t)}^2dt
    \end{align*}
    \item If $0<P<\infty$, $x(t)$ called a \term{power signal}: \begin{align*}
        P=\lim_{T\to\infty}\frac{1}{2T}\int_{-T}^T\abs{x(t)}^2dt
    \end{align*}
\end{enumerate}

\newp
\textbf{Complex signals}: $z(t)=x(t)+jy(t)$ [$x=\Re(z),y\Im(z)$ real signals] \begin{itemize}
    \item \term{Euler's formula}: $e^{j\phi}=\cos\phi+j\sin\phi$
    \item Phasor representation: $z=x+jy=r\cdot e^{j\phi}$ [$r=\sqrt{x^2+y^2},\;\phi=\arctan(\frac{y}{x})$]
    \item Complex relations: complex conjugate $z=x+jy\implies z^\ast=x-jy$ \begin{itemize}
        \item Modulus/magnitude: $\abs{z^2}=zz^\ast$; inverse: $-j=\frac{1}{j}$
    \end{itemize}
    \item Important relations: \begin{itemize}
        \item $\cos\theta=\frac{1}{2}[e^{j\theta}+e^{-j\theta}]$; \quad$\sin\theta=\frac{1}{2j}[e^{j\theta}-e^{-j\theta}]$
    \end{itemize}
\end{itemize}


\begin{whitebox}
    \ulbf{Signal Models}
    \begin{enumerate}
        \item Real sinusoid/cosine: $x(t)=A\cos(\omega t-\theta)$
        \item Complex sinusoid: $x(t)=Ae^{j(\omega t+\theta)}=A\cos(\omega t+\theta)+jA\sin(\omega t+\theta)$ \begin{itemize}
            \item Real part drawn as solid line; imaginary part as dotted line (shifted $\pi/2$ from real)
        \end{itemize}
        \item Exponential signal: $x(t)=e^{\sigma t}$ [$\sigma>/<0\implies$ exponential growth/decay]
        \item Damped/growing sinusoid: $x(t)=e^{\sigma t}\cos(\omega t+\theta)$
        \item Complex exponential: $x(t)=e^{(\sigma+j\omega)t}=e^{\sigma t}e^{j\omega t}$ \begin{itemize}
            \item Can plot $\sigma,\omega$ on $x$ \& $y$ axes; rep. rates of growth/decay, oscillation
        \end{itemize}
        \item \term{Unit heavyside/step function}: $u(t)=1_{t\geq0}+0\cdot1_{t<0}$ 
        \item \term{Unit rectangle} - two definitions: \begin{align*}
            \text{rect}(t)=\begin{cases}
                1 & \abs{t}\leq\frac{1}{2} \\ 0 & \text{otherwise}
            \end{cases};\qquad\text{rect}_{\Delta}(t)=\begin{cases}
                \frac{1}{\Delta} & \abs{t}\leq\frac{\Delta}{2} \\ 0 & \text{otherwise}
            \end{cases}
        \end{align*} \begin{itemize}
            \item Has area 1 for all $\Delta$
        \end{itemize}
        \item \term{Unit ramp/relu}: $r(t)=t\cdot 1_{t\geq0}+0\cdot1_{t<0}=t\cdot u(t)$
        \item \term{Unit triangle} [area 1]: \begin{align*}
            \Delta(t)=\begin{cases}
                1-\abs{t} & \abs{t}<1 \\ 0 & \text{else}
            \end{cases}
        \end{align*}
        \item \term{Dirac function/delta}: $\delta(t)=\infty\cdot 1_{t=0}+0\cdot1_{t\neq0}$
    \end{enumerate}
\end{whitebox}

\newp
\ulbf{Dirac Delta} \begin{itemize}
    \item Has area 1; intuitively, $\delta(t)=\lim_{\Delta\to0}\text{rect}_{\Delta}(t)$
    \item Properties: \begin{enumerate}
        \item \term{Impulse sampling property}: $x(t)\cdot\delta(t)=x(0)\cdot\delta(t)$ [area $x(0)$] \begin{itemize}
            \item Shifting: $x(t)\delta(t-T)=x(T)\delta(t-T)$
        \end{itemize}
        \item \term{Impulse sifting property}: $\int_{-\infty}^\infty x(t)\delta(t)dt=x(0)$ \begin{itemize}
            \item Can integrate from $-\infty$ to $0^{-}$ [doesn't include delta] or $0^{+}$ [includes delta]
        \end{itemize}
        \item Impulse \& unit step: $\int_{-\infty}^t\delta(\tau)d\tau=u(t)$; $\frac{du(t)}{dt}=\delta(t)$
        \item Scaling: $\delta(bt)=\frac{1}{b}\delta(t)$
    \end{enumerate}
    \item Graphing: Draw as an arrow at $t=0$ [or $t=T$], write area if area is $x(0)$
\end{itemize}


\pagebreak
\section{Systems}
\textit{Systems} transform input signals $x(t)$ into output signals $y(t)$ \begin{enumerate}
    \item Scaling system: $x(t)\mapsto y(t)=ax(t)$; draw as triangle with an $a$
    \item Differentiator: $x(t)\mapsto x'(t)$; rectangle with $\frac{d}{dt}$
    \item Integrator: $x(t)\mapsto \int_a^tx(\tau)d\tau$ [$a=0$ or $\infty$]; rectangle with $\int$
    \item Squarer: $x(t)\mapsto x(t)^2$; rectangle with $(\cdot)^2$
    \item Systems with multiple inputs: \begin{itemize}
        \item Summing system: $x_1,x_2\mapsto x_1+x_2$; circle with $+$
        \item Difference: Draw + next to $x_1$, - next to $x_2$
        \item Multiplier: circle with $\times$
    \end{itemize}
\end{enumerate}

\newp
\begin{whitebox}
    \ulbf{System Properties} \begin{itemize}
        \item System called \term{BIBO stable} if a bounded input results in a bounded output \begin{itemize}
            \item Bounded signal: $\exists\;M$ constant s.t. $\abs{x(t)}\leq M_x<\infty\;\forall\;t$
        \end{itemize}
        \item System called \term{causal} if it only uses values of input signal $x(t)$ for $t\leq0$
        \item System called \term{time-invariant} if $[\mathcal{S}(x(t-\alpha))](t)=[\mathcal{S}(x(t))](t-\alpha)$
        \item System is called \term{linear} if the following hold: \begin{enumerate}
            \item \term{Homogeneity/scaling}: For any signal $x$ and scalar $a$, $S(ax)=aS(x)$
            \item \term{Superposition/additivity}: For any signals $x_1$ and $x_2$, $S(x_1+x_2)=S(x_1)+S(x_2)$
        \end{enumerate}
        \item System called \term{LTI} if it is both linear and time-invariant \begin{itemize}
            \item Can freely swap order of LTI systems
            \item Note: differentiation is an LTI system
        \end{itemize}
        \item System has \term{memory} if its output depends on past or future values of the input; otherwise, called \term{memoryless}
        \item System called \term{invertible} if $\exists\;\mathcal{S}^{inv}$ s.t. $x(t)=\mathcal{S}^{inv}(\mathcal{S}(x(t)))$
    \end{itemize}
\end{whitebox}


\pagebreak
\section{Impulse Response}
Can define responses of a system $H$ to several different signals: \begin{enumerate}
    \item \term{Zero response}: $H(0)$
    \item \term{Impulse response}: $h(t)=H(\delta(t))$
    \item \term{Step response}: $H(u(t))$ \begin{itemize}
        \item Note: $\frac{d}{dt}H(u(t))=H(\delta(t))$
    \end{itemize}
\end{enumerate}

\newp
Define the \term{convolution integral}: \begin{eqnbox}
    f*g=\int_{-\infty}^\infty f(\tau)g(t-\tau)d\tau
\end{eqnbox}
$\implies$ for any LTI system $H$ with impulse response $h(t)=H(\delta(t))$, input signal $x(t)$: \begin{eqnbox}
    y(t)=(x*h)(t)=\int_{-\infty}^\infty x(\tau)h(t-\tau)d\tau)
\end{eqnbox}

\begin{whitebox}
    \ulbf{Properties of Convolution}: \begin{enumerate}
        \item \textbf{Commutativity}: $(x*h)(t)=(h*x)(t)$ \quad[via change of vars $\gamma=t-\tau$]
        \item \textbf{Associativity}: $(f*(g*h))(t)=((f*g)*h)(t)$ \quad[via swapping order of integration]
        \item \textbf{Distributivity}: $f*(g+h)=f*g+f*h$
        \item \textbf{LTI}: Convolution systems are \ul{linear \& time-invariant} \begin{itemize}
            \item Cascade/composition: $y=(x*f)*g=x*h$ [$h=f*g$]
            \item $h(t)=\frac{d}{dt}s(t)$
        \end{itemize}
    \end{enumerate}
\end{whitebox}

\newp
\textbf{Other properties}: \begin{itemize}
    \item[($\ast$)] BIBO stability: $H$ is BIBO-stable $\Longleftrightarrow$ $h(t)$ is absolutely integrable ($\int_{-\infty}^\infty\abs{h(t)}dt\in\reals$ finite)
    \item[($\ast$)] Delay element of convolution: $x(t)*\delta(t)=x(t)$; $x(t)*\delta(t-t_d)=x(t-t_d)$
    \item[($\ast$)] Causal systems: $h(t)=0$ for $t<0$
    \item[($\ast$)] Integrating via convolution: $x(t)*u(t)=\int_{-\infty}^tx(t)dt$
\end{itemize}


\pagebreak
\section{Fourier Series}
Let $f(t)$ cts. periodic signal with fundamental period $T_0\implies$ can write as: \begin{eqnbox}
    f(t)=\sum_{k=-\infty}^\infty c_ke^{jk\omega_0t}
\end{eqnbox}
where $\omega_0=\frac{2\pi}{T_0}$, and $c_k$ are Fourier coefficients of $f(t)$ given by: \begin{eqnbox}
    c_k=\frac{1}{T_0}\int_{\tau}^{\tau+T_0}f(t)e^{-jk\omega_0t}dt
\end{eqnbox} \begin{itemize}
    \item For non-continuous signals, equality for $f(t)$ fails at discontinuities
    \item $c_0$ is the time-averaged mean of signal
\end{itemize}

\newp
A signal $x(t)$ is an \term{eigenfunction} of system $\mathcal{S}$ if $\mathcal{S}(x(t))=ax(t)$ for some constant eigenval. $a\in\complex$. \begin{itemize}
    \item Complex exponentials $e^{j\omega t}$ are eigenfunctions of LTI systems: $a=re^{j\theta}\implies ax(t)=re^{j(\omega t+\theta)}$
    \item Disprove: via example, or looking at $y(t)$
\end{itemize}

\newp
Amplitude \& phase: can plot $A/\phi$ vs. $\omega$, corresponding to signal's Fourier series at that $\omega$ \begin{itemize}
    \item Signal perfectly determined by amplitude, pphase spectrums
\end{itemize}

\newp
Note: sinc$(t)=\frac{\sin(\pi t)}{t}$

~\\ \newp
\ulbf{Fourier Symmetries}: \begin{itemize}
    \item $f(t)$ real $\implies$ $c_k^\ast=c_{-k}:$ \quad$\Re(c_k)=\Re(c_{-k})$, $\Im(c_k)=-\Im(c_{-k})$
    \item $\abs{c_k}=\abs{c_{-k}}$
    \item $\angle c_k=-\angle c_{-k}$ $\left[\angle c_k=\arctan(\frac{\Im(c_k)}{\Re(c_k)})\right]$
    \item Also: \begin{itemize}
        \item $x(t)$ even \& real $\implies$ $c_k$ are real
        \item $x(t)$ odd \& real $\implies$ $c_k$ are imaginary
    \end{itemize}
\end{itemize}

\newp
\term{Parseval's theorem}: Let $x(t)=\sum_{k=-\infty}^\infty c_ke^{jk\omega_0t}$, then power $P=\sum_{k=-\infty}^\infty\abs{c_k^2}$

\pagebreak
\section{Fourier Transforms}
\ulbf{Fourier Transform \& Inverse FT of a signal $f(t)$} [$t\to\omega\Leftrightarrow\omega\to t$]: \begin{eqnbox}
    F(j\omega)=\int_{-\infty}^\infty f(t)e^{-j\omega t}dt\quad\Longleftrightarrow\quad f(t)=\frac{1}{2\pi}\int_{-\infty}^\infty F(j\omega)e^{j\omega t}d\omega
\end{eqnbox}

\pstart
\textbf{Existence of the FT}: Sufficient (but not necessary) condition: \begin{eqnbox}
    \int_{-\infty}^\infty\abs{f(t)}dt<\infty
\end{eqnbox}
\begin{whitebox}
    \ulbf{Fourier Properties}: \begin{enumerate}
        \item \textbf{LTI}: The Fourier transform system $\fourier{\cdot}$ is LTI
        \item \textbf{Complex conjugate}: $f^\ast(t)\Longleftrightarrow F^\ast(-j\omega)$
        \item \textbf{Time scaling}: $\fourier{f(at)}=\frac{1}{\abs{a}}F\left(j\frac{\omega}{a}\right)$
        \item \textbf{Time shifting}: $\fourier{f(t-\tau)}=e^{-j\omega\tau}\fourier{f(t)}$
        \item \textbf{Duality}: $\fourier{F(t)}=2\pi f(-j\omega)$
        \item \textbf{Convolution theorem}: For two arbitrary signals $f_1(t),f_2(t)$: \begin{eqnbox}
            \fourier{(f_1*f_2)(t)}=F_1(j\omega)F_2(j\omega)
        \end{eqnbox}
        \item \textbf{Parseval's Theorem}: $\int_{-\infty}^\infty\abs{f(t)}^2dt=\frac{1}{2\pi}\int_{-\infty}^\infty\abs{F(j\omega}^2d\omega$
        \item \textbf{Derivative}: $\fourier{f'(t)}=j\omega F(j\omega)$ \begin{itemize}
            \item Dual: $(-jt)f(t)=F'(j\omega)$
        \end{itemize}
        \item \textbf{Modulation}: $\fourier{f(t)e^{j\omega_0t}}=F(j(\omega-\omega_0))$ \begin{itemize}
            \item $\fourier{f(t)\cos(\omega_0t)}=\frac{1}{2}(F(j(\omega-\omega_0))+F(j(\omega+\omega_0)))$
            \item $\fourier{f(t)\sin(\omega_0t)}=\frac{1}{2j}(F(j(\omega-\omega_0))-F(j(\omega+\omega_0)))$
        \end{itemize}
    \end{enumerate}
\end{whitebox}

\newp
\ulbf{FT Symmetries/Properties}: \begin{itemize}
    \item For any $f(t)$ real/imag/complex: $f(t)$ even $\implies$ $F(j\omega)$ even [same for odd] \begin{itemize}
        \item Even $\implies$ mag., phase even;
    \end{itemize}
    \item Real signals have Hermitian FT (Hermitian symmetry): $f(t)$ real $\implies$ $F(-j\omega)=F^\ast(j\omega)$ \begin{itemize}
        \item $\abs{X(j\omega)}$ even; $\angle X(j\omega)=-\angle X(-j\omega)$
    \end{itemize}
    \item Imaginary signals have anti-Hermitian FT: $f(t)$ imaginary $\implies$ $F(-j\omega)=-F^\ast(j\omega)$
    \item Complex signals have neither Hermitian nor anti-Hermitian FT
\end{itemize}

\iffalse
\begin{minipage}[t]{\textwidth}
    \centering
    \includegraphics[width=0.7\linewidth]{ft_pairs.png}
\end{minipage}
\fi

\pagebreak
\section{Frequency Response}
\term{Frequency response}: Fourier transform of impulse response $\underline{h(t)\Longleftrightarrow H(j\omega)}$ \begin{eqnbox}
    y(t)=h(t)*x(t)\Longleftrightarrow Y(j\omega)=H(j\omega)X(j\omega)
\end{eqnbox}
\begin{itemize}
    \item $H(j\omega)$ also called \term{transfer function}: characterizes how input changed at every frequency
\end{itemize}

\newp
\ulbf{Filters} \begin{enumerate}
    \item \textbf{Low-pass filters}: $=0\;\forall\;\omega\not\in[-\omega_c,\omega_c]$ [=1]

    \textit{Ideal low-pass}: [not causal, infinitely long $h(t)$] \begin{align*}
        H(j\omega)=\rect{\omega/(2\omega_c)}\Longleftrightarrow h(t)=\frac{\omega_c}{\pi}\sinc\left(\frac{\omega_ct}{\pi}\right)
    \end{align*}
    \item \textbf{High-pass filters}: $=0\;\forall\;\omega\in[-\omega_c,\omega_c]$ [=1]

    \textit{Ideal high-pass}: \begin{align*}
        H(j\omega)=1-\rect{\omega/(2\omega_c)}\Longleftrightarrow h(t)=\delta(t)-\frac{\omega_c}{\pi}\sinc\left(\frac{\omega_ct}{\pi}\right)
    \end{align*}
    \item \textbf{Band-pass filters}: $=0\;\forall\;\omega\not\in[\pm\omega_0-\omega_c,\pm\omega_0+\omega_c]$ [=1]

    \textit{Ideal band-pass:} \begin{align*}
        H(j\omega)=\rect{(\omega+\omega_0)/(2\omega_c)}+\rect{(\omega-\omega_0)/(2\omega_c)}\Longleftrightarrow h(t)=2\cos(\omega_0t)\cdot\left[\frac{\omega_c}{\pi}\sinc\left(\frac{\omega_ct}{\pi}\right)\right]
    \end{align*}
\end{enumerate}
\newp
Causal filters: can truncate \& shift [disortionless]

\newp
\term{Distortionless (LTI) system}: Output is only a shifted/scaled version of the input $y(t)=Kx(t-t_d)$ \begin{itemize}
    \item \textit{Frequency response}: $Y(j\omega)=Ke^{-j\omega t_d}\cdot X(j\omega)\implies H(j\omega)=Ke^{-j\omega t_d}$ 
    
    (Namely: $\abs{H(j\omega)}=K$, $\angle H(j\omega)=-\omega t_d$)
\end{itemize}

\newp
\term{Group delay} - indicates how much a signal will be delayed (constant for LTI) \begin{align*}
    t_d(\omega)=-\frac{d}{d\omega}\angle H(j\omega)
\end{align*}

\pagebreak
\section{Sampling}
\underline{\textbf{Impulse train (\& FT)} [$\omega_0=2\pi/T$]}: \begin{eqnbox}
    \delta_T(t)=\sum_{k=-\infty}^\infty\delta(t-kT)\quad\Longleftrightarrow\quad F(j\omega)=\omega_0\sum_{k=-\infty}^\infty\delta\left(x-k\omega_0\right)
\end{eqnbox}
\ulbf{Impulse train sampling}: \begin{eqnbox}
    f(t)\delta_T(t)=\sum_{k=-\infty}^\infty f(kT)\cdot\delta(t-kT)\quad\Longleftrightarrow\quad \Tilde{F}(j\omega)=\frac{1}{T}\sum_{k=-\infty}^\infty F(j(\omega-k\omega_0))
\end{eqnbox}

\pstart
\ulbf{Nyquist Sampling Theorem} \\[6pt]
Find $B=\frac{1}{2\pi}\abs{\omega_{\max}}$ [maximum frequency in Hz] $\implies$ to perfectly recover a signal, require: \begin{eqnbox}
    \omega_0\geq4\pi B\;[=2\omega_{\max}]\Longleftrightarrow\frac{1}{T}=f_{sample}\geq 2B\quad\text{[\textit{Nyquist rate}]}
\end{eqnbox} \begin{itemize}
    \item $\omega_{\max}$: largest $\abs{\omega}$ with $F(\pm j\omega)\neq0$
    \item \textit{Nyquist interval}: $T=1/(2B)$
    \item To recover signal: find $\Tilde{F}(j\omega)=\frac{1}{T}\sum_{k=-\infty}^\infty F(j(\omega-k\omega_0))$ \& use LPF to find central copy
    \item $\omega_0<4\pi B\implies$ \ulbf{aliasing}: modulated copies start to overlap \begin{itemize}
        \item Taking LPF will reconstruct a different, lower-frequency signal
        \item \textit{Anti-aliasing filter}: Use a LPF on signal beore sampling to reduce aliasing (causes distortion)
    \end{itemize}
\end{itemize}
\newp
\ulbf{Interpolation}
\begin{enumerate}
    \item Zero-order hold: At each $t$, take the value of the last-measured signal/sample
    \item Linear interpolation: Connect sampled points with a line
    \item Perfect interpolation: Returns the function exactly

    Under Nyquist rate $\implies$ use \textbf{Whittaker-Shannon interpolation formula} [IFT of $\Tilde{F}(j\omega)H_{\text{LPF}}$]: \begin{eqnbox}
        f(t)=\sum_{k=-\infty}^\infty f(kT)\;\sinc(2Bt-k)
    \end{eqnbox}
\end{enumerate}

\pagebreak
\section{The Laplace Transform}
\underline{\textbf{Unilateral Laplace Transform} for a causal signal $f(t)u(t)$}: \begin{eqnbox}
    F(s)=\int_{0^-}^\infty f(t)e^{-st}dt
\end{eqnbox}
\begin{itemize}
    \item \textbf{Region of convergence/ROC}: Range of values $(\sigma,\omega)$ [$s=\sigma+j\omega$] for which $F(s)$ converges
\end{itemize}
\begin{whitebox}
    \ulbf{Laplace Transform Properties}
    \begin{enumerate}
        \item The Laplace transform $\laplace[\cdot]$ is \textbf{linear}
        \item \textbf{Time scaling} [$a>0$]: $\laplace[f(at)]=\frac{1}{a}F\left(\frac{s}{a}\right)$
        \item \textbf{Time shift} [$T>0$]: $\laplace[f(t-T)]=e^{-sT}F(s)$
        \item \textbf{Frequency shift}: $\laplace[f(t)e^{s_0t}]=F(s-s_0)$
        \item \textbf{Convolution Theorem}: $\laplace[f_1(t)*f_2(t)]=F_1(s)F_2(s)$
        \item \textbf{Integration}: $\laplace\left[\int_0^t f(\tau)d\tau\right]=\frac{1}{s}F(s)$
        \item \textbf{Derivative}: $\laplace[f'(t)]=sF(s)-f(0)$
        \item \textbf{Multiplication by $t$}: $\laplace[tf(t)]=-F'(s)$
    \end{enumerate}
\end{whitebox}
\textbf{Fourier \& Laplace transforms}: For cases where ROC includes $s=j\omega$ axis: \begin{eqnbox}
    F(j\omega)=F(s)\vert_{s=j\omega}
\end{eqnbox}

\pstart
\textbf{Laplace transform \& Diff. Eqs.}: Differentiating a signal multiplies LT by $s$; integrating multiplies by $1/s$ \begin{itemize}
    \item Differential equations in time domain $\Leftrightarrow$ algebraic equations in Laplace domain
\end{itemize}

\newp
\ulbf{Partial Fraction Expansion} \\[6pt]
Can find poles $\lambda_1,\hdots,\lambda_n$ of $F(s)$ \& residues $r_1,\hdots,r_n$ s.t.: \begin{eqnbox}
    F(s)=\frac{b(s)}{a(s)}=\frac{b_ms^m+\hdots+b_1s+b_0}{a_ns^n+\hdots+a_1s+a}\implies F(s)=\frac{r_1}{s-\lambda_1}+\hdots+\frac{r_n}{s-\lambda_n}
\end{eqnbox}
\begin{itemize}
    \item Inverse Laplace transform [for $t\geq0$]: \begin{eqnbox}
        f(t)=\laplace^{-1}[F(s)]=\laplace^{-1}\left[\frac{r_1}{s-\lambda_1}+\hdots+\frac{r_n}{s-\lambda_n}\right]=r_1e^{-\lambda_1t}+\hdots+r_ne^{-\lambda_nt}
    \end{eqnbox}
    \item \textbf{Repeated roots}: \begin{gather*}
        \text{$(s-\lambda^k)$ a root}\implies\text{Include a term }\frac{r_{1,i}}{(s-\lambda)^i}\text{ for each $i=1,\hdots,k$} \\
        \longrightarrow\quad\laplace^{-1}\left[\frac{r}{(s-\lambda)^k}\right]=\frac{r}{(k-1)!}t^{k-1}e^{\lambda t}
    \end{gather*}
\end{itemize}

\newp
\ulbf{Finding Partial Fraction Expansions} \\[6pt]
(i) \textbf{Cover-up procedure} [primary method]: \begin{eqnbox}
    r_k=(s-\lambda_k)F(s)\bigg\vert_{s=\lambda_k}
\end{eqnbox}
\begin{itemize}
    \item \textbf{Repeated roots}: $r_{1,k}$ via cover-up (multiplying by $(s-\lambda)^k$); for $r_{1,k-j}$ [$j\neq0$]: \begin{eqnbox}
        r_{1,k-j}=\frac{1}{j!}\frac{d^j}{ds^j}(F(s)(s-\lambda)^k)\bigg\vert_{s=\lambda}
    \end{eqnbox}
\end{itemize}

\pstart
(ii) \textbf{L'Hopital's Rule}: Can use that \begin{eqnbox}
    F(s)=\frac{b(s)}{a(s)}\implies r_k=\frac{b(\lambda_k)}{a'(\lambda_k}
\end{eqnbox}

\pstart
(iii) \textbf{Quadratic Factors}: Take partial fraction expansion directly \begin{eqnbox}
    F(s)=\frac{r_1s+r_2}{as^2+bs+c}\implies\text{ ILT via $e^{-at}\cos(\omega t)$ Laplace pair}
\end{eqnbox}

\newp
($\ast$) \textbf{Nonproper Rational Functions} [Degree $m$ of numerator $\geq$ degree $n$ of denominator] \\[8pt]
Split into polynomial $+$ proper rational function: \begin{eqnbox}
    F(s)=\underbrace{\frac{b(s)}{a(s)}}_{\text{nonproper}}\longrightarrow F(s)=\underbrace{c(s)}_{\text{polynom.}}+\underbrace{\frac{d(s)}{a(s)}}_{\text{proper}}
\end{eqnbox}
\begin{itemize}
    \item $d(s)/a(s)$ proper $\to$ use partial fraction expansion
    \item Obtain $c(s)$ via polynomial long division [stop when order of subtracted $<$ order of denominator] \begin{align*}
        \text{ILT of $c(s)$: }\underline{c(s)=c_0+c_1s+\hdots+c_{m-n}s^{m-n}\Longleftrightarrow c_0\delta(t)+c_1\delta^{(1)}(t)+\hdots+c_{m-n}\delta^{(m-n)}(t)}
    \end{align*}
\end{itemize}

\end{document}
