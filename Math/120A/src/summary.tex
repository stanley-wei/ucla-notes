\documentclass[13pt]{extarticle}

\usepackage[export]{adjustbox}
\usepackage{amsmath}
\usepackage{amssymb}
\usepackage{amsthm}
\usepackage{empheq}
\usepackage{fancyhdr}
\usepackage{graphicx}
\usepackage{soul}
\usepackage{subcaption}
\usepackage{subfiles}
\usepackage[most]{tcolorbox}
\usepackage[explicit]{titlesec}
\usepackage{ulem}

\newenvironment{whitebox}{\begin{tcolorbox}[colback=white]}{\end{tcolorbox}}
\newenvironment{defnbox}{\begin{tcolorbox}[colback=blue!3!white]\begin{definition}}{\end{definition}\end{tcolorbox}}
\newenvironment{thmbox}{\begin{tcolorbox}[colback=blue!3!white]\begin{theorem}}{\end{theorem}\end{tcolorbox}}


\newtcbox{\emphbox}[1][]{%
    nobeforeafter, math upper, tcbox raise base,
    enhanced, colframe=black,
    colback=white, boxrule=1pt,
    #1}
\NewDocumentEnvironment{eqnbox}{b}{}{
  \begin{empheq}[box=\emphbox]{equation*}
    #1
  \end{empheq}
}
\NewDocumentEnvironment{alignbox}{b}{}{
  \begin{empheq}[box=\emphbox]{align*}
    #1
  \end{empheq}
}
\NewDocumentEnvironment{gatherbox}{b}{}{
  \begin{empheq}[box=\emphbox]{gather*}
    #1
  \end{empheq}
}


\theoremstyle{definition}
\newtheorem*{corollary}{Corollary}
\newtheorem*{definition}{Definition}
\newtheorem*{lemma}{Lemma}
\newtheorem*{prop}{Prop}
\newtheorem*{recall}{Recall}
\newtheorem*{theorem}{Theorem}
\newtheorem*{thm}{Thm}

\theoremstyle{remark}
\newtheorem*{note}{Note}
\newtheorem*{notation}{Notation}

\newcommand{\pstart}[0]{\noindent}
\newcommand{\newp}[0]{\vspace{8pt}\pstart}
\newcommand{\term}[1]{\textit{\textbf{#1}}}
\newcommand{\ulbf}[1]{\textbf{\ul{#1}}}

\newcommand{\titleul}[1]{\newp\ulbf{#1}\vspace{7pt}~\\}
\newcommand{\ctrrule}[2]{\begin{center}\rule{#1}{#2}\end{center}}


% General Commands
\newcommand{\abs}[1]{\left|#1\right|}
\newcommand{\mbold}[1]{\boldsymbol{#1}}
\newcommand{\image}[1]{\text{Im}\left(#1\right)}
\newcommand{\set}[1]{\left\{#1\right\}}
\newcommand{\simplies}[0]{\;\Rightarrow\;}

\newcommand{\fourier}[1]{\mathcal{F}\left[#1\right]}
\newcommand{\fourierinv}[1]{\mathcal{F}^{-1}\left[#1\right]}

% Fields
\newcommand{\reals}[0]{\mathbb{R}}
\newcommand{\complex}[0]{\mathbb{C}}
\newcommand{\integers}[0]{\mathbb{Z}}
\newcommand{\naturals}[0]{\mathbb{N}}

% Linear Algebra Commands
\newcommand{\inner}[2]{\left<#1,#2\right>}
\newcommand{\norm}[1]{\left|\left|#1\right|\right|}
\newcommand{\rank}[1]{\text{rank}\left(#1\right)}
\newcommand{\trace}[1]{\text{tr}(#1)}
\newcommand{\vect}[1]{\left<#1\right>}
\newcommand{\vspan}[1]{\text{span}\left\{#1\right\}}

% Calculus Commands
\newcommand{\grad}[0]{\nabla}

% \usepackage[lmargin=0.3in,rmargin=0.3in,bmargin=0.3in,tmargin=0.3in]{geometry} % <- Shrunken sizing
\usepackage[lmargin=0.7in,rmargin=0.7in,tmargin=0.7in,bmargin=0.7in]{geometry}
\pagenumbering{gobble}

\titleformat{\section}{\large\bfseries}{\thesection}{1em}{#1\hrule\vspace*{-14pt}}


\begin{document}

% \pstart Stanley Wei

\begin{center}
    \begin{Large}
        \textbf{Math 120A: Differential Geometry}
    \end{Large}
    
    \begin{large}
        \vspace{8pt}
        Prof. K. Varvarezos $\vert$ Fall 2024
    \end{large}
\end{center}
\tableofcontents

\pagebreak
\section{Curves and Parametrizations}
\begin{tcolorbox}[colback=white]
    \begin{definition}
        A \term{parametrized curve} is a function $\gamma:I\to\reals^n$ (where $I$ is  a potentially unbounded interval $I\subseteq\reals$). \begin{itemize}
            \item Say that $\gamma$ is \term{smooth} if $\gamma$ is (i) continuous and (ii) infinitely differentiable
            \item Say that $\gamma$ is \term{regular} if its $1^{st}$ derivative is never-vanishing, i.e. $\dot\gamma\neq\Vec{0}\;\forall\;t\in I$
        \end{itemize}
    \end{definition}
\end{tcolorbox}

\vspace{1pt}
\begin{definition}
    A set $C\subseteq\reals^n$ is called a \term{(smooth) curve} if it is (locally) the image of a smooth, regular parametrized curve. \begin{itemize}
        \item Smoothness is a \ul{local} property!
        \item Can still have multiple connected components, discontinuities
    \end{itemize}
\end{definition}

\newp
\subsection{Tangents to Curves}
Let $\gamma:(a,b)\to\reals^n$ be a smooth parametrized curve: \\[-6pt] \begin{itemize}
    \item The vector $\dot\gamma(t)=\frac{d}{dt}\gamma(t)$ is tangent to the curve $C=\image(\gamma)$ at point $\gamma(t)$
    \item The \term{speed} at time $t$ is given by $\norm{\dot\gamma(t)}$
    \item The \term{arc length} of $\gamma$ is given by: \begin{eqnbox}
        l(\gamma)=\int_a^b\norm{\dot\gamma(t)}dt
    \end{eqnbox}
\end{itemize}

\begin{prop}
    If a parametrized curve $\gamma$ is \term{constant speed} (i.e. $\norm{\dot\gamma(t)}=c\;\forall\;t\in I$ for some $c\in\reals$), then $\dot\gamma(t)\perp\ddot\gamma(t)\;\forall\;t\in I$.
\end{prop}

\newp
\subsection{Reparametrization}
\begin{prop}
    Let $\gamma_1:(a_1,b_1)\to\reals^n, \gamma_2:(a_2,b_2)\to\reals^n$ be smooth and regular parametrizations of the same curve $C$. Then for each $t\in(a_1,b_1)$, $\exists$ intervals $(c_1,d_1)\subseteq(a_1,b_1),$ and $(c_2,d_2)\subseteq(a_2,b_2)$ [with $t\in(c_1,d_1)$] and a smooth bijection [with smooth inverse] $\rho:(c_1,d_1)\to(c_2,d_2)$ s.t. \begin{align*}
        \implies\text{$\underline{\gamma_1=\gamma_2\circ\rho}$ on the interval $(c_1,d_1)$}
    \end{align*}
\end{prop}

\newp
\subsection{Level Sets}
\begin{tcolorbox}[colback=white]
    \begin{definition}
        Let $f$ be a (smooth) function $f:\reals^n\to\reals$, $c\in\reals$; then the \term{level set} of $f$ at height $c$ is defined by: \begin{eqnbox}
            f^{-1}(c)=\set{x\in\reals^n:f(x)=c}
        \end{eqnbox}
    \end{definition}

    \begin{center}
        \vspace{3pt}
        \rule{14cm}{0.4pt}
    \end{center}

    \begin{definition}
        Given a function $f$ as described above: \begin{itemize}
            \item A point $x\in\reals^n$ is called a \term{critical point} of $f$ if $\grad f(x)=0$
            \item A scalar $c\in\reals$ is called a \term{critical value} of $f$ if $f(x)=c$ for some critical point $x\in\reals^n$; otherwise, $c$ is called a \term{regular value}
        \end{itemize}
    \end{definition}
\end{tcolorbox}

\begin{thm}
    Let $f:\reals^n\to\reals$; then for any regular value $c\in\reals$ of $f$, $f^{-1}(c)$ is a (smooth) curve.
\end{thm}


\pagebreak
\section{Vector Fields}
\textbf{Notation}:
\begin{itemize}
    \item \textit{Base-pointed vectors}: Let $p,v\in\reals^n\to$ denote the vector $v$ based at point $p$ as $(p,v)$
    \item Also denote the vector space of vectors $\reals^n$ based at $p$ as $\reals_p^n$ \begin{itemize}
        \item Base point $p$ unchanged under addition, scalar multiplication
    \end{itemize}
\end{itemize}

\newp
\begin{tcolorbox}[colback=white]
    \begin{definition}
        Let $U\subseteq\reals^n$. A \term{(smooth) vector field} on $U$ is a smooth function $X:U\to U\times\reals^n$ such that $\forall\;p\in U,X(p)=(p,v)$ for some $v\in\reals^n$.
    \end{definition}
\end{tcolorbox}
\begin{note}
    For any curve $C$ with parametrization $\gamma$, can define its \term{tangent vector field} $X:C\to C\times\reals^2$ by $X(\gamma(t))=(\gamma(t),\dot\gamma(t))$.
\end{note}

\pstart
\begin{notation}
    Given a parametrized curve $\gamma:(a,b)\to\reals^n$, a vector field along $\gamma$ is a function $X:(a,b)\to\reals^n\times\reals^n$ s.t. $X(t)=(\gamma(t),v(t))$ [for some smooth function $v:(a,b)\to\reals^n$]. \begin{itemize}
        \item Can take derivatives of $X$ w.r.t. $t$: $\dot X(t)=(\gamma(t),\dot v(t))$
    \end{itemize}
\end{notation}

\begin{definition}
    Given a smooth function $f:\reals^n\to\reals$, the \term{gradient vector field} $\grad f$ of $f$ is defined by ($\forall\;p\in\reals^n$): \begin{eqnbox}
        \grad f(p)=\left(p,\left(\frac{\partial f}{\partial x_1}(p),\hdots,\frac{\partial f}{\partial x_n}(p)\right)\right)
    \end{eqnbox} \begin{itemize}
        \item For any parametrization $\gamma$ of a level set $f^{-1}(c)$ [$c$ regular value] - $\grad f(\gamma(t))\perp\dot\gamma(t)$
    \end{itemize}
\end{definition}

\newp
\subsection{Integral Curves}
\begin{definition}
    Let $X:U\to U\times\reals^2$ be a (smooth) vector field. An \term{integral curve} of $X$ is a parametrization $\gamma:(a,b)\to U$ s.t. \begin{eqnbox}
        X(\gamma(t))=(\gamma(t),\dot\gamma(t))\;\forall\;t\in(a,b)
    \end{eqnbox} 
\end{definition}

~\\[-18pt]
\begin{thm}
    Given $X:U\to U\times\reals^2$ and $p\in U$, $\exists$ a ``maximal'' integral curve of $X$ through $p$ - $\gamma:(a,b)\to U$ s.t. (assuming WLOG that $a<0<b$): \begin{enumerate}
        \item $\gamma(0)=p$
        \item If $\beta:I\to U$ is another integral curve of $X$ with $\beta(0)=p$, then $I\subseteq(a,b)$ and $\beta(t)=\gamma(t)\;\forall\;t\in I$
    \end{enumerate}
\end{thm}

\newp
\subsection{Tangent Spaces}
\begin{definition}
    Let $C\subseteq\reals^n$ be a (smooth) curve, and let $p\in C$. A vector $(p,v)$ based at $p$ is said to be \term{tangent} to $C$ if $v=\dot\gamma(t)$ for some (smooth, not necessarily regular) parametrization $\gamma$ of $C$ with $\gamma(t)=p$. \\[-4pt]

    $\rightarrow$ The \term{tangent space} of $C$ at $p$ is defined by: \begin{eqnbox}
        T_pC=\set{(p,v):\text{$(p,v)$ is tangent to $C$}}
    \end{eqnbox}
\end{definition}

\begin{prop}
    Let $C$ be a smooth curve and $\gamma$ a (smooth, regular) parametrization of $C$ with $\gamma(t)=p$; then: \begin{eqnbox}
        T_pC=\vspan{(p,\dot\gamma(t))}
    \end{eqnbox}
    In particular, $T_pC$ is a 1D subspace of $\reals_p^n$.
\end{prop}

\newp
\subsection{Orientation}
\begin{definition}
    Let $\gamma:(a,b)\to\reals^n$ be a parametrized curve, and let $X(t)=(\gamma(t),v(t))$ be a vector field along $\gamma$. \begin{itemize}
        \item $X$ is called a \term{unit vector field} if $\norm{v(t)}=1\;\forall\;t\in(a,b)$
        \item $X$ is called a \term{normal vector field} if $v(t)$ is orthogonal to $\dot\gamma(t)\;\forall\;t\in(a,b)$
    \end{itemize}
\end{definition}

\newp
\textit{Obtaining orientation} [$\reals^2$]: \begin{enumerate}
    \item If $C=f^{-1}(c)$ is a level set of a function $f$, take $\grad f/\norm{\grad f}$
    \item If $\gamma$ parametrizes $C$, find $N(t)$ by rotating $\dot\gamma(t)$ CCW $\pi/2$ [$(x,y)\mapsto(-y,x)$]
\end{enumerate}

~\\[-28pt]
\begin{tcolorbox}[colback=white]
    \begin{definition}
        Given a plane curve $C\subseteq\reals^2$, an \term{orientation} of $C$ is a choice of unit normal vector field along $C$.
    \end{definition}
\end{tcolorbox}

~\\[-28pt]
\begin{prop}
    Let $C\subseteq\reals^2$ be a plane curve. If $C$ is connected, then $C$ has exactly 2 orientations.
\end{prop}

\begin{definition}
    For a parametrized plane curve $\gamma:(a,b)\to\reals^2$, say that an orientation $N$ is \term{consistent} with the parametrization if $\forall\;t\in(a,b)$, $N(t)$ is the vector obtained by rotating the tangent direction $\frac{\dot\gamma(t)}{\norm{\dot\gamma(t)}}$ $\pi/2$ counterclockwise, i.e.: \begin{align*}
        N(t)=\frac{(-\dot\gamma_2(t),\dot\gamma_1(t))}{\norm{\dot\gamma(t)}}
    \end{align*}
\end{definition}


\pagebreak
\section{Curvature}
\begin{tcolorbox}[colback=white]
    \begin{definition}
        Let $C\subseteq\reals$ be a simple [i.e. non-self-intersecting] plane curve with orientation $N$. Then the curvature of $C$ at point $p\in C$ is given by: \begin{eqnbox}
            \kappa(p)=\frac{\ddot\gamma(t)\cdot N(\gamma(t))}{\norm{\dot\gamma(t)}^2}
        \end{eqnbox}
        where $\gamma$ is any smooth, regular parametrization of $C$ and $t\in\reals$ s.t. $\gamma(t)=p$.
    \end{definition}
\end{tcolorbox}

\begin{note}
    \begin{itemize}
        \item The curvature of a curve is independent of parametrization $\gamma$.
        \item Inward/outward $\to$ +/-; sharp/shallow $\to$ high/low
    \end{itemize}
\end{note}

\vspace{-4pt}
\begin{definition}
    Let $C\subseteq\reals^2$ be a curve with orientation $N$, and let $p\in C$ s.t. $\kappa(p)\neq0$; then the \term{circle of curvature/osculating circle} of $C$ at $p$ is the circle $C_O\subseteq\reals^2$ with: \begin{enumerate}
        \item Center of curvature $p+\frac{1}{\kappa(p)}N(p)$
        \item Radius of curvature $\frac{1}{\abs{\kappa(p)}}$
        \item Orientation $N_O$ s.t. $N_O(p)=N(p)$
    \end{enumerate}
\end{definition}

\newp
\ulbf{Properties}
\begin{enumerate}
    \item $C_O$ is tangent to $C$ at point $p$
    \item $C_O,C$ have the same curvature at $p$
\end{enumerate}

\newp
\subsection{Frenet-Serret [2D]}
\begin{definition}
    Let $C\subseteq\reals^2$ be a curve with orientation $N$, $\gamma(t)$ a unit-speed parametriztion of $C$

    \pstart
    $\rightarrow$ define the \term{unit tangent vector} to $\gamma$ at time $t$ by: \begin{eqnbox}
        T(t)=(\gamma(t),\dot\gamma(t))
    \end{eqnbox}
\end{definition}

\begin{tcolorbox}[colback=white]
    \ulbf{Frenet-Serret Formulas [2D]} \\[-24pt]

    \begin{minipage}[t]{\textwidth}
        \centering
        \begin{minipage}[t]{0.3\textwidth}
            \centering
            \begin{empheq}[box=\emphbox]{align*}            
                (1)&\; \dot T=\kappa N \\
                (2)&\; \dot N=-\kappa T
            \end{empheq}
        \end{minipage}
        \begin{minipage}[t]{0.1\textwidth}
            \centering
            \vspace{15pt}
            \begin{gather*}
                \Longleftrightarrow
            \end{gather*}
        \end{minipage}
        \begin{minipage}[t]{0.4\textwidth}
            \centering
            \begin{eqnbox}
                \begin{bmatrix}
                    \dot T \\ \dot N
                \end{bmatrix}=\begin{pmatrix}
                    0 & \kappa \\ -\kappa & 0
                \end{pmatrix}\begin{bmatrix}
                    T \\ N
                \end{bmatrix}
            \end{eqnbox}
        \end{minipage}
    \end{minipage}
\end{tcolorbox}

\newp
\subsection{Frenet-Serret [3D]}
Let $\gamma$ be a unit-speed parametrization of a curve $C\subseteq\reals^3$: \begin{enumerate}
    \item Define the \term{unit tangent vector} at $\gamma(t)$ by: \begin{eqnbox}
        T(t)=(\gamma(t),\dot\gamma(t))
    \end{eqnbox}
    \item Define the \term{principal unit normal vector} of $\gamma(t)$ by: \begin{eqnbox}
        N(\gamma(t))=\left(\gamma(t),\frac{\ddot\gamma(t)}{\norm{\ddot\gamma(t)}}\right)
    \end{eqnbox}
    \item Define the \term{binormal unit vector} of $\gamma(t)$ by: \begin{eqnbox}
        B=T\times N
    \end{eqnbox}
\end{enumerate}

\begin{tcolorbox}[colback=white]
    \ulbf{Frenet-Serret Formulas [3D]} \\[-24pt]

    \begin{minipage}[t]{\textwidth}
        \centering
        \begin{minipage}[t]{0.3\textwidth}
            \centering
            \begin{empheq}[box=\emphbox]{align*}            
                (1)&\;\dot T=\kappa N \\
                (2)&\;\dot N=-\kappa T+\tau B \\
                (3)&\;\dot B=-\tau N
            \end{empheq}
        \end{minipage}
        \begin{minipage}[t]{0.1\textwidth}
            \centering
            \vspace{24pt}
            \begin{gather*}
                \Longleftrightarrow\hspace{2pt}
            \end{gather*}
        \end{minipage}
        \begin{minipage}[t]{0.4\textwidth}
            \centering
            \vspace{3pt}
            \begin{eqnbox}
                \begin{bmatrix}
                    \dot T \\ \dot N \\ \dot B
                \end{bmatrix}=\begin{pmatrix}
                    0 & \kappa & 0 \\
                    -\kappa & 0 & \tau \\
                    0 & -\tau & 0
                \end{pmatrix}\begin{bmatrix}
                    T \\ N \\ B
                \end{bmatrix}
            \end{eqnbox}
        \end{minipage}
    \end{minipage}
\end{tcolorbox}
\newp
\textbf{Consequence}: Any curve $C$ completely determined by its initial values + curvature \& torsion.


\pagebreak
\section{Surfaces}
\begin{tcolorbox}[colback=white]
    \begin{definition}
        A \term{surface}/\term{2-manifold} is a set $S\subseteq\reals^n$ s.t. $\forall\;p\in S$, $S$ is locally parametrized in some neighborhood $U$ of $p$ by a smooth function $\phi:D_r^2\to U$ satisfying $\rank(\Jac(\phi))=2$.
    \end{definition}
\end{tcolorbox}

\pstart
\textbf{3 ways to obtain a surface}: \begin{enumerate}
    \item As the image of a local parametrization $S=\image(\phi)$
    \item As the graph of a function $f:\reals^2\to\reals$
    \item As the level set of a smooth function $f:\reals^3\to\reals$
\end{enumerate}

\newp
\textbf{2 notable classes of surfaces}: \begin{enumerate}
    \item Given plane curve $C\subseteq\reals^2$, the \term{cylinder over $C$} is the set $A\subseteq\reals^3$ given by \begin{eqnbox}
        A=\set{(x,y,z):(x,y)\in C}
    \end{eqnbox} \begin{itemize}
        \item $C$ smooth curve $\implies$ cylinder over $C$ is a smooth surface
    \end{itemize}
    \item Given $C\subseteq\reals^2$ curve lying above the x-axis, its \term{surface of revolution} is: \begin{eqnbox}
        S=\set{(x,y,z):\left(x,\sqrt{y^2+z^2}\right)\in C}
    \end{eqnbox}
\end{enumerate}

\newp
\subsection{Tangent Spaces}
\begin{tcolorbox}[colback=white]
    \begin{definition}
        $S\subseteq\reals^3$ a surface, $p\in S$ a point $\implies$ $(p,v)$ is \term{tangent to $S$} if $\exists$ smooth PC $\gamma:(a,b)\to S$ s.t. for some $t_0\in\reals$: \begin{eqnbox}
            \gamma(t_0)=p\,\text{ and }\,\dot\gamma(t_0)=v
        \end{eqnbox}
        $\implies$ The \term{tangent space} of $S$ at $p$ is the set $T_pS\subseteq\reals^3_p$ [$\dim T_pS=2$] defined by: \begin{eqnbox}
            T_pS=\set{(p,v):\text{$(p,v)$ is tangent to $S$}}
        \end{eqnbox}
    \end{definition}
\end{tcolorbox}

\subsection{Orientation for Surfaces}
\begin{tcolorbox}[colback=white]
    \begin{definition}
        Let $S\subseteq\reals^n$ a surface and $p\in S$ $\implies$ the \term{tangent plane} of $S$ at $p$ is: \begin{eqnbox}
            P=\set{p+v:(p,v)\in T_pS}
        \end{eqnbox}
    \end{definition}
    
    \begin{definition}
        A vector $(p,v)$ is \term{orthogonal} to a surface $S$ if it is orthogonal to all vectors $(p,v_0)\in T_pS$
    
        \newp
        $\implies$ Let $S\subseteq\reals^3$ a surface; then an \term{orientation} of $S$ is a choice of unit normal vector field $N:S\to S:\times\reals^3$ s.t. $\forall\;p\in S$: $N(p)\perp S,\norm{N(p)}=1$.
    \end{definition}
\end{tcolorbox}
\begin{itemize}
    \item Note: $S$ a connected surface $\implies$ $S$ has either 0 or 2 orientations
\end{itemize}

\newp
\subsection{The Gauss Map}
\begin{definition}
    Given an oriented curve/surface $A\in\reals^n$ with orientation $N$, the associated \term{Gauss map} is the fuction $G:A\to S^{n-1}$ defined by: \begin{eqnbox}
        N(p)=(p,G(p))\;\forall\;p\in A
    \end{eqnbox} \begin{itemize}
        \item Notation: $S^n$ is unit sphere in $\reals^{n+1}$
    \end{itemize}
\end{definition}

\begin{definition}
    Define the \term{spherical image} of $A$ as the image of the Gauss map of $A$ \begin{itemize}
        \item \textbf{Thm}: If $A=f^{-1}(c)$ compact, then its Gauss map under orientation $N=\pm\frac{\grad f}{\norm{\grad f}}$ is surjective
        \item \textit{Rec}: $A=f^{-1}(c)$ for $f$ smooth $\implies$ $A$ closed; $A\subseteq\reals^n$ compact $\Longleftrightarrow$ $A$ closed \& bounded [Heine-Borel]
    \end{itemize}
\end{definition}

\newp
\subsection{Curves on Surfaces}
\begin{definition}
    Let $S\subseteq\reals^3$ a surface, $P\subseteq\reals^3$, and $p\in S\cap P$ $\implies$ say $P$ is \term{orthogonal} to $S$ at $p$ if $\exists\;(p,v)\neq0$ orthogonal to $S$ and tangent to $P$
\end{definition}


\pagebreak
\section{Curvatures of Surfaces}
\begin{tcolorbox}[colback=white]
    \begin{definition}
        Let $f:\reals^n\to\reals$ smooth, and let $(p,v)\in\reals_p^n$ $\implies$ define \term{derivative of $f$ w.r.t. $(p,v)$} [linear] by (for $\gamma:(a,b)\to\reals^n$ PC w/ $\gamma(t_0)=p,\dot\gamma(t_0)=v$): \begin{eqnbox}
            \grad_{(p,v)}f=\frac{d}{dt}(f\circ\gamma)(t_0)\;[=\grad f(p)\cdot v]
        \end{eqnbox}
        \& f or a vector field $X:U\to U\times\reals^n$: \begin{eqnbox}
            \grad_{(p,v)}X=(X\;\dot\circ\;\gamma)(t_0)\;[=(p,\;(\grad X_1(p)\cdot v,\hdots,\grad X_n(p)\cdot v))]
        \end{eqnbox}
    \end{definition}
\end{tcolorbox}

\begin{tcolorbox}[colback=white]
    \begin{definition}
        Let $S\subseteq\reals^3$ be smooth surface oriented by $N$, and let $p\in S$ $\implies$ define the \term{Weingarten map} of $f$ at $p$ by: \begin{eqnbox}
            L_p(v)=-\grad_{(p,v)}N(p)
        \end{eqnbox}
    \end{definition}    
\end{tcolorbox}

\begin{itemize}
    \item The Weingarten map is linear \& self-adjoint: $L_p(v)\cdot w=L_p(w)\cdot v$
    \item For a curve $\gamma:(a,b)\to S$ on $S$ with $\gamma(t_0)=p,\dot\gamma(t_0)=v$: $\ddot\gamma\cdot N(p)=L_p(v)\cdot v$
    \item Note: $\grad_{(p,v)}N$ only depends on values of $N$ on $S$
\end{itemize}

\newp
\begin{tcolorbox}[colback=white]
    \begin{definition}
        Let $S\subseteq\reals^3$ surface oriented by $N$, and let $p\in S$, $(p,v)\in T_pS$ unit vector. Then the \term{normal curvature} of $S$ at $p$ in the direction $v$ is defined by: \begin{eqnbox}
            k(p,v)=L_p(v)\cdot v
        \end{eqnbox}
    \end{definition}
\end{tcolorbox}

\pstart
\textbf{Corollary}: Let $S\subseteq\reals^3$ oriented surface, $p\in S$, $P$ plane through $p$ orthogonal to $S$ at $p$; then $P\cap S$ is (near $p$) a smooth curve satisfying: \begin{eqnbox}
    \kappa(p)=k(p,v)
\end{eqnbox}

\begin{whitebox}
    \begin{definition}
        Let $S\subseteq\reals^3$ surface oriented by $N$, and let $p\in S$. Let $\set{v_1,v_2}$ be an orthonormal basis (for $T_pS$) of eigenvectors of $L_p$ with eigenvalues $\lambda_1,\lambda_2$; then the principal curvatures of $S$ at $p$ are defined by: \begin{eqnbox}
            k_1(p)=k(p,v_1)=\lambda_1;\quad k_2(p)=k(p,v_2)=\lambda_2
        \end{eqnbox}
    \end{definition}
\end{whitebox}
\begin{itemize}
    \item The principal curvatures $k_1(p)\leq k_2(p)$ are the min \& max of the normal curvatures $k(p,v)$ at $p$: \begin{align*}
        \underline{k_1(p)=\min_vk(p,v);\quad k_2(p)=\max_vk(p,v)}
    \end{align*}
\end{itemize}

\begin{whitebox}
    \begin{definition}
        Let $S\subseteq\reals^3$ an oriented surface, and let $p\in S$. Then the \term{Gauss curvature} of $S$ at $p$ is given by: \begin{eqnbox}
            K(p)=k_1(p)\cdot k_2(p)[=\det L_p]
        \end{eqnbox}
        Similarly, the \term{mean curvature} of $S$ at $p$ is defined by: \begin{eqnbox}
            H(p)=\frac{1}{2}\left(k_1(p)+k_2(p)\right)
        \end{eqnbox}
    \end{definition}
\end{whitebox}
\begin{theorem}
    Let $S\subseteq\reals^3$ a compact oriented surface; then $\exists\;p\in S$ such that either (i) $k(p,v)>0\forall\;v$, or (ii) $k(p,v)<0\;\forall\;v$.

    \newp
    [Equivalently: $K(p)>0$]
\end{theorem}

\begin{lemma}
    Let $S\subseteq\reals^3$ oriented by $N$, and let $X:\reals^3\to\reals^3$ a VF s.t. $X/\norm{X}=N$. Let $p\in S$ \& $\set{v_1,v_2}$ any basis for $T_pS$; then: \begin{align*}
        K(p)=\det\begin{pmatrix}
            \grad_{(p,v_1)}X \\
            \grad_{(p,v_2)}X \\
            X(p)
        \end{pmatrix}\cdot\frac{1}{\norm{X(p)}^2\det\begin{pmatrix}
            v_1 \\ v_2 \\ X(p)
        \end{pmatrix}}
    \end{align*}
\end{lemma}

\newp
\subsection{Fundamental Forms}
\begin{definition}
    Llet $V\subseteq\reals^n$ a subspace of $\reals^n$, and let $T:V\to V$ a self-adjoint linear map. Then the quadratic form associated with $T$ is the map $Q:V\to\reals$ given by: \begin{align*}
        Q(v)=T(v)\cdot v
    \end{align*}
\end{definition}

\begin{whitebox}
    \begin{definition}
        Let $S\subseteq\reals^3$, and let $p\in S$. Then the \term{fundamental forms} of $S$ at $p$ are: \begin{enumerate}
            \item 1$^{st}$ fundamental form: $\ell_p:T_pS\to\reals$ defined by \begin{eqnbox}
                \ell_p(v)=v\cdot v=\norm{v}^2
            \end{eqnbox}
            \item 2$^{nd}$ fundamental form: $\rho_p:T_pS\to\reals$ defined by: \begin{eqnbox}
                \rho_p(v)=L_p(v)\cdot v
            \end{eqnbox}
        \end{enumerate}
    \end{definition}
\end{whitebox}

\newp
\textbf{Notice}: Let $(p,v)\in T_pS,v\neq0$; then: \begin{align*}
    \underline{\rho_p(v)=\norm{v}^2k\left(p,\frac{v}{\norm{v}}\right)}
\end{align*}
In particular, have the following relationships: \begin{enumerate}
    \item $\rho_p$ definite $\Longleftrightarrow$ $K(p)>0$
    \item $\rho_p$ indefinite $\Longleftrightarrow$ $K(p)<0$
    \item $\rho_p$ semi-definite $\Longleftrightarrow$ $K(p)=0$
\end{enumerate}
\newp
[\textbf{Corollary}: Let $S\subseteq\reals^3$ a compact oriented surface; then $\exists\;\rho_p\in S$ with $\rho_p$ definite.]

\newp
\subsection{Isometries}
\begin{definition}
    Let $X_1,X_2\subseteq\reals^n$, and let $\phi:X_1\to X_2$ a diffeomorphism. We call $\phi$ an \term{isometry} if $\forall\;\gamma:(a,b)\to X_1$, the arc length of $\gamma$ is equivalent to the arc length of $\phi\circ\gamma$, i.e.: \begin{align*}
        \underline{\int_a^b\norm{\dot\gamma(t)}dt=\int_a^b\norm{\phi\dot\circ\gamma(t)}dt}
    \end{align*}
\end{definition}

\end{document}
