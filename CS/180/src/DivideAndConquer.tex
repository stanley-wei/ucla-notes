\documentclass[12pt]{extarticle}
\usepackage[export]{adjustbox}
\usepackage{amsmath}
\usepackage{amssymb}
\usepackage{amsthm}
\usepackage{fancyhdr}
\usepackage[lmargin=0.9in,rmargin=0.9in,bmargin=1.25in]{geometry}
\usepackage{graphicx}
\usepackage{soul}
\usepackage{subfiles}
\usepackage[most]{tcolorbox}

\graphicspath{ {./images/} }

\newtheorem{theorem}{Theorem}[section]
\newtheorem{corollary}{Corollary}[theorem]
\newtheorem{lemma}[theorem]{Lemma}

\theoremstyle{definition}
\newtheorem*{definition}{Definition}
\newtheorem*{problem}{Problem}
\newtheorem{case}{\textbf{Case}}

\theoremstyle{remark}
\newtheorem*{remark}{Remark}
\newtheorem*{observation}{Observation}

\newcommand{\probname}[1]{\noindent \textbf{\textit{#1}}}
\newcommand{\probtitle}[1]{\noindent \textbf{\ul{#1}}}
\newcommand{\claim}[1]{\noindent Claim: \textit{#1}}

\begin{document}
\subsection*{Overview (Divide \& Conquer)}
\begin{center}
\textbf{Principle} (\textbf{\textit{Divide and Conquer}}): \textit{If a problem is too difficult to solve directly, we can instead solve it by partitioning it into more solvable sub-problems. }

\vspace{10pt}
\rule{100mm}{0.4pt}
\end{center}

\noindent \textit{Divide \& Conquer}:
\begin{itemize}
    \item A common approach is to utilize \ul{recursive partitioning}.
    \item Divide-and-conquer solutions actually exist for many [most?] problems, though they are not always the optimal solution to the problem.
\end{itemize}

\subsection{Merge Sort}
If an array is too long to sort directly, then we can instead sort it by dividing it into subarrays that can be sorted directly.

\begin{tcolorbox}[colback=blue!50!red!13!white]
    \probtitle{Algorithm (Merge Sort)}
    \begin{enumerate}
        \item Divide the array into some number of subarrays of equal size. \begin{itemize}
            \item Divide subarrays into even smaller subarrays as needed. Keep dividing until the divided subarrays are small enough to sort directly. 
        \end{itemize}
        \item Sort all subarrays.
        \item \textit{Merging step}: Once each subarray is sorted, then we can keep combining (merge) our subarrays into larger subarrays until we have returned to the original array.\begin{itemize}
            \item \textit{Combination}: We can merge two subarrays into a new subarray by continuously picking the minimum of the next element for each subarray.
        \end{itemize}
    \end{enumerate}
\end{tcolorbox}

\subsubsection*{Time Complexity}
\textit{Merging Step}: Given two arrays of size $p$, $q$, each element of the output array is added in O($1$) $\to$ total runtime: O($p+q$)\\

\noindent\textit{Overall Runtime}: Let $T(n)$ be the time required to sort $n$ elements. Per the algorithm: \begin{enumerate}
    \item $T(n)=2T(\frac{n}{2})+O(n)$
    \item $T(1)$ = O($1$)
\end{enumerate}
\begin{center}
    We can define the cost of merging $n$ elements to be $\lambda n$, for some $\lambda $.
\end{center}
\begin{align*}
    T(n)&=2T(\frac{n}{2})+\lambda n\\
    &=2(2T(\frac{n}{2})+\frac{1}{2}\lambda n)+\lambda n\\
    &=2^2\cdot T(\frac{n}{2^2})+2\cdot \lambda n\\
    &=2^i\cdot T(\frac{n}{2^i})+i\cdot \lambda n\,\text{(by induction)}
\end{align*}
After $\log_2(n)$ recursions, $T(\frac{n}{2^i})$ becomes just $T(1)$:
\begin{align*}
    \implies T(n)&=2^{\log_2(n)}T(\frac{n}{2^{\log_2n}})+2\lambda \cdot n\log_2(n)\\
    &=n\cdot T(1)+2\lambda \cdot n\log_2(n)
\end{align*}
\begin{gather*}
    \implies\text{O($n\log n$)}
\end{gather*}

\subsubsection*{Notes}
\begin{itemize}
    \item The \textbf{\textit{master theorem}} is a general formula for finding the time complexities of recursive divide-and-conquer algorithms \textit{\`{a} la} merge sort.
    \item It is not immediately obvious that O($n\log n$) is optimal, since the most obvious lower bound for the time complexity of sorting is O($n$); however, it has been proven that O($n\log n$) \textit{is}, in fact, optimal.
\end{itemize}

\pagebreak
\subsection{Counting Inversions}
\begin{problem}[\probname{Crossing Problem}]
    Given two parallel lines between which there are total $n$ lines (potentially overlapping), find an algorithm to determine the number of crossings.\begin{itemize}
        \item \textit{Trivial solution}: compare every possible pair of lines [O($n^2$)]
    \end{itemize}
\end{problem}

\begin{observation}
    We can assign numerical indices $1,2,\hdots$ to each line, in order of the coordinates where those lines touch one of the two parallel axes (e.g. the bottom axis). Then, two lines $i,j$ cross iff $i$ is before $j$ on one axis and after $j$ on the other. \begin{itemize}
        \item The number of crossings is equal to the number of out-of-order indices (\textit{inversions}).
    \end{itemize}
\end{observation}

\begin{center}
    \adjincludegraphics[trim={0 0 0 0},clip,scale=0.4]{Images/Notes/counting_inversions.png}
\end{center}

\subsubsection*{Algorithm (Counting Inversions)}
\begin{enumerate}
    \item Assign each line a label based on where that line meets the bottom axis, as described above. Create a counter [initially 0] to hold the number of crossings.
    \item Take the sequence in which the lines meet the top axis, and perform merge sort.

    During every merge: \begin{enumerate}
        \item Case 1: The number on the right is larger than the number on the left; then the two lines did not cross.\begin{itemize}
            \item We use ``left'' to denote the subsequence that occurred earlier in the original sequence (of the two subsequences being compared)
        \end{itemize}
        \item Case 2: The number on the right is smaller than the number on the left; then the number on the right crossed the current left number, and every number in the left array larger than the current left number.

        We increment the crossings counter accordingly.
    \end{enumerate}
\end{enumerate}

$\implies$ \textbf{Runtime}: O($n\log n$) for merge sort

\pagebreak
\subsection{Closest Pair Problem}
\begin{problem}[\probname{Closest Pair}]
    Given set of $n$ points in $\mathbb{R}^i$, we want to find the closest pair of points, i.e. two points $x,y$ [$x\neq y$] such that $||x-y||\leq ||x'-y'||$ for any other pair $x',y'$. \begin{itemize}
        \item \textit{Trivial solution}: compare every possible pair of points [O($n^2$)]
        \item \textit{Note}: There are a number of possible metrics for quantifying distance in $\mathbb{R}^i$ (e.g. the Manhattan/L1 norm); in this case, we consider only the \textit{Euclidean/L2} norm.
    \end{itemize}
\end{problem}

\noindent In the case of $\mathbb{R}^1$ [$i=1$], we can obtain an O($n\log n$) solution simply by sorting the points by coordinate. However, simple greedy approaches (e.g. projection onto an axis, taking polar coordinates) fail in general $\mathbb{R}^n$.

\begin{observation}
    If we partition the set of points $S$ into two subsets $S_1,S_2$, then the closest pair of points in $S$ is either a closest pair in one of $S_1,S_2$, or is a pair $(x\in S_1,y\in S_2)$.
    
    \vspace{6pt}
    Namely, if we define $\delta$ to be the minimum of the distance between closest pairs in $S_1,S_2$, a pair $(x\in S_1,y\in S_2)$ is a closest pair in $S$ only if $||x-y||\leq\delta$. \begin{itemize}
        \item Suggests a divide-and-conquer approach: we can find the closest pair in $S_1,S_2$ recursively, then look for closest pairs between partitions during the merging step. 

        $\to$ We can obtain an O($n\log n$) runtime if we can merge in O($n$).
    \end{itemize}
\end{observation}

\subsubsection*{The Merging Step}
We can use $(i-1)$-dimensional subspaces of $\mathbb{R}^i$ (e.g. 2D planes in $\mathbb{R}^3$) as dividers between $S_1,S_2$. Then a pair of points between partitions can only be less than $\delta$ apart if both points are less than $\delta$ away from the divider.

Taking $\mathbb{R}^2$ as an example: if our divider is a vertical line at $x=\lambda$, then a pair of points $a\in S_1,b\in S_2$ can only have $||a-b||<\delta$ if $|a_x-\lambda|,|b_x-\lambda|<\delta$.

However, we also observe that $a,b$ must also have that $|a_y-b_y|<\delta$.\\

\noindent \textit{Consequence}: For any point $p\in S_1$, a point $p'\in S_2$ can have $||p'-p||<\delta$ only if: 
\begin{enumerate}
    \item $p_x'\in(\lambda,\lambda+\delta)$
    \item $p_y'\in(p_y-\delta,p_y+\delta)$
\end{enumerate}

Then any closest pair partner for $p$ must be within the $\delta\times 2\delta$ rectangular region $(\lambda,\lambda+\delta)\times(p_y-\delta,p_y+\delta)$, i.e. we need only look at the points in that region. Then we can find that the merging step is O($n$) if we can show that the number of points within this rectangle is bounded above by a constant.

\begin{proof}
    We observe that the rectangle is strictly contained in $S_2$.
    
    We can divide the rectangle into 8 squares of size $\frac{\delta}{2}\times\frac{\delta}{2}$. We see that all of these squares are themselves strictly in $S_2$; then there is at most one point in each square. \begin{itemize}
        \item Recall the definition for $\delta$: if there were more than one point in a square, then we could find a pair in $S_2$ less than $\delta$ apart.
    \end{itemize}

    Since there is at most one point in each square, then there are at most 8 points in the rectangle; then for any point $p\in S_1$, we need only consider at most 8 points $p'\in S_2$ to identify all possible closest pairs. 
\end{proof}

~\\
\noindent Then, if we can find a way to make sure we look at all said points (if they exist) in O(1), we can merge in O($n$).

For simplicity, assume the points belong to $\mathbb{R}^2$, and that the divider is some vertical line $x=x_0$. Then the region of eligible points is the set of points within the $x$-interval $(x_0-\delta,x_0+\delta$; and for any point $p=(x_1,y_1)\in S_1$, we need only look at the 8 points in $S_2$ with y-coordinates nearest to $y_1$ (that fall within the $x$-interval).

\begin{center}
    \adjincludegraphics[trim={0 0 0 0},clip,scale=0.4]{Images/Notes/closest_pair.png}
\end{center}

Then, if the points are sorted by y-coordinate, we can check our 8 points in O($1$) by simply looking at the 8 points before/after $p$; then it remains to find a way to have all points by sorted by y-coordinate, while limiting our merges to O($n$) runtime.

\vspace{8pt}
\noindent $\implies$\textbf{Solution}: Sort all points by y-coordinate ahead of time, and merge in a way that preserves sortedness. \begin{itemize}
    \item During merging: assume left and right are sorted ahead of time, then use the same process as merge sort to merge.
\end{itemize}

\begin{tcolorbox}[colback=blue!50!red!7!white]
    \probtitle{Algorithm (Closest Pair - $\mathbb{R}^2$)}
    \begin{enumerate}
        \item Sort the set of points by y-coordinate.
        \item Keep using vertical lines to divide the set of points in half. 
        \item Keep dividing subsets into additional subsets, until subsets are of size $1$ or $2$. 
        \item Compute the closest pair within each subset (if a pair exists).
        \item \textit{Merging step}: Once the closest pair in each subset is found, then we can keep merging our subsets into larger subsets until we have returned to the original set.

        For each merge:
        \begin{enumerate}
            \item Merge points in sorted order (by y-coordinate). \begin{itemize}
                \item During the merge: keep track of points of $S_1$, $S_2$ within the $2\delta$ $x$-interval. For each point in $S_1$, look at the 8 points of $S_2$ above/below it and find the minimum distance. Keep track of the closest such pair.
            \end{itemize}
            \item Set the closest pair in the merged set to be the closest of: \begin{enumerate}
                \item The closest pair in $S_1$.
                \item The closest pair in $S_2$.
                \item The closest pair between $S_1$, $S_2$.
            \end{enumerate}
        \end{enumerate}
    \end{enumerate}
    \vspace{5pt}
    \noindent\textbf{Runtime}: O($n\log n$)
\end{tcolorbox}

\end{document}
