\documentclass[12pt]{extarticle}
\usepackage[export]{adjustbox}
\usepackage{amsmath}
\usepackage{amssymb}
\usepackage{amsthm}
\usepackage{fancyhdr}
\usepackage[lmargin=0.9in,rmargin=0.9in,bmargin=1.25in]{geometry}
\usepackage{graphicx}
\usepackage{soul}
\usepackage{subfiles}
\usepackage[most]{tcolorbox}

\graphicspath{ {./images/} }

\newtheorem{theorem}{Theorem}[section]
\newtheorem{corollary}{Corollary}[theorem]
\newtheorem{lemma}[theorem]{Lemma}

\theoremstyle{definition}
\newtheorem*{definition}{Definition}
\newtheorem*{problem}{Problem}
\newtheorem{case}{\textbf{Case}}

\theoremstyle{remark}
\newtheorem*{remark}{Remark}
\newtheorem*{observation}{Observation}

\newcommand{\probname}[1]{\noindent \textbf{\textit{#1}}}
\newcommand{\probtitle}[1]{\noindent \textbf{\ul{#1}}}
\newcommand{\claim}[1]{\noindent Claim: \textit{#1}}

\begin{document}
\subsection{Introduction to Graphs}

\noindent\ul{\textbf{Graphs}}
\begin{definition}
    A \textbf{graph} is composed of \textbf{nodes} and \textbf{edges} between nodes. \begin{itemize}
        \item A \textbf{\textit{digraph}} is a graph where each edge has an associated direction.
        \item A \textbf{\textit{weighted graph}} is a graph where each edge has an associated weight.
        \item\textit{Notation}: $n$ nodes, $m$ edges \begin{itemize}
            \item Given a graph with $n$ nodes, the maximum number of possible edges is $\frac{(n)(n-1)}{2}\sim n^2$.
        \end{itemize}
    \end{itemize}
\end{definition}

\begin{center}
    \adjincludegraphics[trim={0 0 0 0},clip,scale=0.25]{Images/Notes/paths.png}
\end{center}

\begin{definition}
    A \textbf{path} is a sequence of nodes/vertices $v_1$, ..., $v_k$ of a graph, where for any $v_n$, $v_{n+1}$ there exists an edge ($v_n$, $v_{n+1}$). \begin{itemize}
        \item A \textit{simple path} is a path where all vertices are distinct.
        \item A \textit{cycle} is a path where the endpoint is the same as the start point.
        \item The \textbf{distance} between two nodes $u,v$ is the length of the shortest path $u\to v$.
        \item A graph is called \textbf{connected} if, for any two vertices in the graph, there exists a path between them.
    \end{itemize}
\end{definition}

~\\
\noindent\ul{\textbf{Trees}}
\begin{definition}
    A graph is called a \textbf{tree} if it is connected and does not contain a cycle. \begin{itemize}
        \item Nodes in a tree are referred to as \textit{descendants} of a root node, with according \textit{parent/child} relationships between adjacent nodes in the tree.
    \end{itemize}
\end{definition}
\noindent \textbf{Properties of Trees}: \begin{enumerate}
    \item An $n$-node tree has exactly $n-1$ edges
    \item A tree is a maximum network without a cycle.
\end{enumerate}

~\\
\noindent\ul{\textbf{Representations of Graphs}}
\begin{enumerate}
    \item \textbf{Adjacency Matrix}: Represents an $n$-node graph as an $n\times n$ matrix, where the $i,j$ cell is 1 if $\exists$ an edge $(v_i,v_j)$ in the graph, 0 otherwise. \begin{itemize}
        \item Suitable for \textit{dense graphs}, where nodes generally have close to $n$ neighbors.\begin{itemize}
            \item \textbf{Advantage}: Checking whether two given nodes are adjacent is a fixed O(1). 
            \item\textbf{Disadvantage}: Given a node, the cost of finding all adjacent nodes is a fixed O($n$) due to needing to scan an entire row of the array.
        \end{itemize}
        \item In weighted graphs, the value of a cell $i,j$ is set to the \textit{weight} of the edge $(v_i,v_j)$, if it exists.
    \end{itemize}
    \item\textbf{Adjacency List}: Represents an $n$-node graph via $n$ node objects, where each node contains a linked list of all edges containing that node. \begin{itemize}
        \item Suitable for \textit{sparse graphs}, where nodes generally have much less than $n$ neighbors. \begin{itemize}
        \item\textbf{Advantage}: Given a node $x$, the cost of finding all nodes adjacent to $x$ is only directly proportional to the number of neighbors of $x$, not to $n$.
        \item\textbf{Disadvantage}: Given a node $x$, the cost of checking if it is adjacent to a node $y$ is proportional to the number of neighbors of $x$.
        \end{itemize}
        \item In digraphs, a node may have two linked lists for holding incoming and outgoing edges, respectively.
    \end{itemize}
\end{enumerate}

\noindent In both cases: the complexity of representing a graph with $n$ vertices, $e$ edges is \ul{O($n+e$)}.

\pagebreak
\subsection{Graph Traversal: BFS \& DFS}
Two main algorithms for traversing unweighted graphs: \textbf{BFS} and \textbf{DFS}.

\begin{tcolorbox}[colback=green!8!white]
\textbf{\ul{Breadth-First Search (BFS)}}: \textit{“Explore things that are close together first”}
\\[-9pt]

Given a graph $G(V,E)$:
\begin{enumerate}
    \item Begin at a root node $u$. Call the set $\{u\}$ ``\textit{layer 0}''.
    \item From the root node: find all adjacent nodes [\textit{\textbf{neighbors}}] in $G$. Call the set of neighbors of $u$, ``\textit{layer 1}''.
    \item Once all neighbors of $u$ have been found, find all neighbors of neighbors of $u$ not already in layers 0/1. Call this set ``\textit{layer 2}''. Continue finding layers 3, 4, etc. until all nodes have been added to a layer.
\end{enumerate}

\begin{center}
    \adjincludegraphics[trim={0 0 {.51\width} 0},clip,scale=0.2]{Images/Notes/bfs_dfs.png}
\end{center}

\probtitle{BFS Trees}
\begin{definition}[\probname{BFS Tree}]
    During BFS, we call every edge used when first visiting a node, a \textbf{tree edge}. A subgraph of $G$ composed of only tree edges from an instance of BFS [rooted at $u$] is called a \textbf{BFS tree} [rooted at $u$].
\end{definition}
\begin{definition}[\probname{Layer}]
    The \textbf{layer} of a node $v$ in a BFS tree rooted at $u$ is defined as the distance from the root node $u$ to $v$ in the tree, and is equivalent to the length of the shortest path $u\to v$ in $G$.
\end{definition}
\end{tcolorbox}

\begin{tcolorbox}[colback=blue!8!white]
\textbf{\ul{Depth-First Search (DFS)}}: “\textit{Explore everything in one direction first}”
\\[-9pt]

Given a graph $G(V,E)$:
\begin{enumerate}
    \item Begin at a root node $u$.
    \item From the root node: pick a random neighbor of $u$, and visit it.
    \item Upon visiting a node: visit any neighboring node that has not already been visited. Continue doing so until we reach a node with no unvisited neighbors, then backtrack until we reach a node with unvisited neighbors.
\end{enumerate}

\begin{center}
    \adjincludegraphics[trim={{.5\width} 0 0 0},clip,scale=0.2]{Images/Notes/bfs_dfs.png}
\end{center}

\probtitle{DFS Trees}
\begin{definition}[\probname{DFS Tree}]
    A subgraph of $G$ composed of only the edges taken from an instance of DFS is called a \textbf{DFS tree}.\begin{itemize}
        \item Any edges in $G$ not in the DFS tree, form a cycle with the edges of the DFS tree.
    \end{itemize}
\end{definition}
\end{tcolorbox}

\subsubsection*{BFS vs DFS}
BFS, DFS are ``\textit{orthogonal algorithms}'': \begin{itemize}
    \item \textbf{BFS}: Used for finding \textbf{distances} from a node $u$ to other nodes in the graph
    \item \textbf{DFS}: Used for finding \textbf{cycles} in a graph
\end{itemize}

\subsubsection*{Implementations}
\begin{tcolorbox}[colback=green!8!white]
\textbf{\ul{Breadth-First Search (Implementation)}}
\\[-9pt]

Given a graph $G(V,E)$:
\begin{enumerate}
    \item Begin with a root node $u$. Initialize a \textbf{queue} [LIFO] containing unvisited nodes. Add all neighbors of the root node to the queue.
    \item While the queue is nonempty: pop a node $v$ from the queue. For every edge $(v,w)$ with $v$ as an endpoint: if the node $w$ has not yet been visited, add $w$ to the queue and record the edge $(v,w)$.
\end{enumerate}
\end{tcolorbox}

\begin{tcolorbox}[colback=blue!8!white]
\textbf{\ul{Depth-First Search (Implementation)}}
\\[-9pt]

Given a graph $G(V,E)$:
\begin{enumerate}
    \item Begin at a root node $u$. Initialize a \textbf{stack} [FIFO] containing unvisited nodes. Add all neighbors of the root node to the stack.
    \item While the stack is nonempty: pop a node $v$ from the stack. For every edge $(v,w)$ with $v$ as an endpoint: if the node $w$ has not yet been visited, add $w$ to the stack and record the edge $(v,w)$.
\end{enumerate}
\end{tcolorbox}

\subsubsection*{Time Complexity}
For both BFS, DFS: every vertex is added to the stack/queue exactly once [O($V$)], during the first visit to that vertex. Similarly, every edge is seen exactly twice (once on each endpoint) $\rightarrow$ O($E$). Taken together: O($V+E$).\\

\noindent There are two methods of interpreting this complexity:
\begin{enumerate}
    \item \textbf{\textit{Linear}}: O($V+E$) is linear with respect to $V$, $E$.
    \item \textbf{\textit{Exponential}}: The maximum number of edges in a graph is dependent on $V$; namely, proportional to $V^2$. If we treat $E$ as dependent on $V$: O($V+E$) $\rightarrow$ O($V+V^2$) $\rightarrow$ O($V^2$).
\end{enumerate}

\noindent Given a graph with $n$ nodes, $m$ edges: complexity is typically said to be O($m+n$), linear with respect to the input size.

\newpage
\subsection{Extensions of BFS/DFS}
\subsubsection{Graph Coloring}
\begin{definition}[\probname{Bipartite graph}]
     A \textbf{bipartite graph} [\textit{2-colorable graph}] is a graph where the nodes can be partitioned into two sets $X$ and $Y$ such that every edge in the graph is between a node in $X$ and a node in $Y$. \begin{itemize}
        \item Equivalently: A bipartite graph is a graph with \textbf{no odd cycles}.
         \item Components of a graph may be individually referred to bipartite/not bipartite.
     \end{itemize}
\end{definition}

\begin{problem}[\probname{Testing Bipartiteness}]
    Given a graph $G=(V,E)$ [undirected] with one component, determine whether $G$ is bipartite [2-colorable].
\end{problem}

\begin{center}
    \adjincludegraphics[trim={0 0 0 0},clip,scale=0.3]{Images/Notes/bipartite_graphs.png}
\end{center}

\noindent \textbf{Approach}: Assign each vertex one of two colors (ex: red or blue). If the graph is bipartite, then we must be able to find an assignment such that given a red vertex, all of its neighbors are blue, and vice versa.

\begin{tcolorbox}[colback=red!10!white]
    \probtitle{Algorithm (Graph Coloring)}
    \begin{enumerate}
        \item Assign a random node to be the root node of a BFS, and color it red.
        \item Use BFS to generate a BFS tree of the graph.
        \item Assign all nodes in every odd layer to be blue, and all nodes in every even layer to be red.
        \item The graph is bipartite if and only if there are no edges between two nodes of the same color (i.e. in the same layer).
    \end{enumerate}
    \noindent\textbf{Runtime}: O($m+n$) [BFS]
\end{tcolorbox}

\pagebreak
\subsubsection{Strong Connectedness}
\begin{definition}
    A \textbf{digraph} $G$ is called “\textbf{strongly connected}” if, for any two nodes $u$ and $v$ in $G$, there exists a path from each to the other ($u$ and $v$ are \textit{mutually reachable}).
\end{definition}
\begin{problem}[\probname{Strong Connectivity}]
    Determine if a digraph G is strongly connected.
\end{problem}
\begin{tcolorbox}[colback=white!90!black]
    \probtitle{Algorithm (Strong Connectivity)}
    \begin{enumerate}
        \item Pick an arbitrary vertex $u\in G$; from $u$, run BFS on $G$. If every node is included in the resulting BFS tree, then there is a path from $u$ to every node in the tree.
        \item Look at the graph $G_{rev}$: $G$, but with the direction of all edges reversed. Run BFS from $u$ on $G_{rev}$. If every node is included in the resulting BFS tree, then there is a path from every node in the tree to $u$.
        \item If there is a path from $u$ to every node, and a path from every node to $u$, then $G$ is strongly connected.
    \end{enumerate}
    \noindent\textbf{Runtime}: O($m+n$) for BFS
\end{tcolorbox}

\pagebreak
\subsection{Topological Sorting}
\subsubsection*{Digraph Terminology}
\begin{definition}[\probname{Degree}]
    The in-degree and out-degree of a node $u$ in a digraph are the number of edges into and out of $u$, respectively. \begin{itemize}
        \item A node with an in-degree of 0 is called a \textbf{\textit{source}}.
        \item A node with an out-degree of 0 is called a \textbf{\textit{sink}}.
    \end{itemize}
\end{definition}

\begin{figure}[h]
    \centering
    \adjincludegraphics[trim={0 {0.1\height} 0 {0.15\height}},clip,scale=0.7]{Images/Notes/digraph_degree.png}
\end{figure}

\begin{definition}
    A \textbf{directed acrylic graph} (\textbf{DAG}) is a directed graph with no cycles.
\end{definition}
\begin{remark}
    Any DAG must have at least one source node.
\end{remark}
\begin{observation}
    Removing nodes and/or edges from a DAG, preserves direct acylicity.
\end{observation}

\subsubsection*{Topological Ordering}
\begin{definition}
    A \textbf{topological ordering} of a digraph $G$ is an ordering $v_1,v_2,...$ of nodes of $G$ such that for any edge $v_i\to v_j$ in $G$, $j>i$. \begin{itemize} 
        \item A topological ordering need not be unique.
    \end{itemize}
\end{definition}

\begin{problem}[\probname{Topological Sorting}]
    Given a directed graph $G(V, E)$, find a topological ordering of $G$ [or determine that none exists].
\end{problem}

\begin{observation}
    A graph $G$ has a topological ordering only if $G$ has no cycles, i.e. if $G$ is a DAG.
\end{observation}

\begin{tcolorbox}[colback=blue!75!red!10!white]
    \probtitle{Algorithm (Topological Sorting)}
    \begin{enumerate}
        \item Initialize an empty topological ordering.
        \item While not all nodes have been added to the ordering: \begin{enumerate}
            \item Pick any source node $u$, and append $u$ to our ordering. \begin{itemize}
                \item If there is no source node, then $G$ is not a DAG; then we terminate.
            \end{itemize}
            \item Delete $u$ from $G$, as well as any edges with $u$ as an endpoint.\begin{itemize}
                \item If $G$ is a DAG, then $G$ will still be a DAG after removing $u$.
            \end{itemize}
        \end{enumerate}
    \end{enumerate}
\end{tcolorbox}

\subsubsection*{Implementation}
Given a graph with $n$ vertices, $e$ edges, we can pre-compute the in-, out-degree of all nodes before beginning the topological ordering algorithm. 

This can be done through a loop through the edges of the graph in O($e$) time.\\

\noindent During the topological sorting process, source nodes can be stored using a stack/queue. \begin{itemize}
    \item Before sorting, we can find all sources [nodes with an in-degree of 0] in the original graph by scanning all vertices in linear time [O($n$)].
    \item At each step, we pop a source from the queue to add to our ordering.
    \item When deleting an edge: \begin{itemize}
        \item We can decrement the in-degree of the node on end of the edge by 1.
        \item If this causes the in-degree of the node to become 0, then we add it to our queue.
    \end{itemize}
\end{itemize}
The time complexity of the loop is thus O($n+e$): each node is popped from the queue exactly once [O($n$)], and each edge is looked at/removed exactly once [O($e$)].\\

\noindent$\rightarrow$ \textbf{Overall runtime}: O($n+e$)



\end{document}