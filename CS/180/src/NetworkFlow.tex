\documentclass[12pt]{extarticle}
\usepackage[export]{adjustbox}
\usepackage{amsmath}
\usepackage{amssymb}
\usepackage{amsthm}
\usepackage{fancyhdr}
\usepackage[lmargin=0.9in,rmargin=0.9in,bmargin=1.25in]{geometry}
\usepackage{graphicx}
\usepackage{soul}
\usepackage{subfiles}
\usepackage[most]{tcolorbox}

\graphicspath{ {./images/} }


\newtheorem{theorem}{Theorem}[section]
\newtheorem{corollary}{Corollary}[theorem]
\newtheorem{lemma}[theorem]{Lemma}

\theoremstyle{definition}
\newtheorem*{definition}{Definition}
\newtheorem*{problem}{Problem}
\newtheorem{case}{\textbf{Case}}

\theoremstyle{remark}
\newtheorem*{remark}{Remark}
\newtheorem*{observation}{Observation}

\newcommand{\probname}[1]{\noindent \textbf{\textit{#1}}}
\newcommand{\probtitle}[1]{\noindent \textbf{\ul{#1}}}
\newcommand{\claim}[1]{\noindent Claim: \textit{#1}}

\begin{document}
\subsubsection*{Introduction to Networks}
\begin{definition}[\probname{Network}]
    A \textbf{network} is a weighted directed graph where each edge has an associated \textbf{\textit{capacity}} [weight]. \begin{itemize}
        \item An \textbf{S-T network} is a network with a specified starting vertex S, end vertex T.
        \item The capacity of an edge may be used to indicate the number of items that can be sent through that edge.
        \item Cycles are allowed.
    \end{itemize}
\end{definition}

\begin{center}
    \adjincludegraphics[trim={0 {0.12\height} 0 {0.23\height}},clip,scale=0.8]{Images/Notes/network_flow.png}
\end{center}

\begin{definition}[\probname{Flow}]
    Given an S-T network, a \textbf{flow} is a function mapping each edge to a non-negative number. A \textbf{\textit{legal flow}} is a flow such that the number assigned to each edge is less than or equal to the capacity of the edge.
\end{definition}
\begin{definition}[\probname{Max Flow}]
    Given a flow F on an S-T network, we defined the \textbf{\textit{capacity}} \textbar F\textbar\ of F to be the sum of the flow to T. We define a \textbf{max flow} on F to be a legal flow with maximum capacity.
\end{definition}

\subsection{Max Flow Problem}
\begin{problem}[\probname{Max Flow/Min Cut}]
    Given an S-T network, find a max flow on the network.\\[5pt]
    \noindent \textbf{\textit{Rules}}: \begin{enumerate}
        \item The flow through any edge is bounded by the capacity of the edge, i.e. $f_i\leq c_i\;\forall\;i$.
        \item Edge capacities are \textit{positive nonzero integers}.
        \item \textit{Conservation of flow}: The flow into any vertex needs to be equal to the flow out of it ($\sum_{\to v_i}f_j$=$\sum_{v_i\to}f_j$) \begin{itemize}
            \item Exceptions: $S$, $T$
        \end{itemize}
    \end{enumerate}
\end{problem}

\subsubsection*{Solutions to Max Flow}
\noindent Simple case: if the network is a single path, then the max flow is simply the minimum of capacities in the network.\\

\noindent \textit{\textbf{Greedy principle}}: We can keep looking for paths $S\to T$ that do not yet contain any saturated edges [edges where flow = capacity] and saturating them, until no more paths exist.\\

\noindent \textit{Issue}: Random path selection does not always provide a max flow, but $\nexists$ any criteria for path selection that would provide a max flow on every network.

\noindent\textit{Solution}: We can modify greedy to allow it to ``change its mind''.\begin{itemize}
    \item \textit{Mechanism}: Anytime the greedy algorithm picks an edge $(u\to v)$ and runs $x$ amount of flow through it, we can create a \textit{back edge} $(v\to u)$ with capacity $x$. \begin{itemize}
        \item Graph with back edges called an \textit{augmented graph/network} or \textit{residual network}.
    \end{itemize}
    \item Can repeat the greedy process on the augmented graph to find ``augmented paths'' $S\to T$, stopping when no more augmented paths exist. \begin{itemize}
        \item \textit{Interpretation}: Sending flow through a back edge $(v\to u)$ is equivalent to decreasing the flow sent through the original edge $(u\to v)$.
    \end{itemize}
\end{itemize}

\begin{tcolorbox}[colback=yellow!10!white]
    \probtitle{Algorithm (Ford-Fulkerson)}
    \begin{enumerate}
        \item Begin by finding any path $S\to T$ in the graph, e.g. via BFS/DFS.
        \item ``Augment'' the graph: \begin{enumerate}
            \item For each edge $(u\to v)$ on the path, increment the flow on that edge by 1 and create a \textit{back edge} $(v\to u)$ with capacity 1. \begin{itemize}
                \item If the edge $(v\to u)$ already exists, increase its capacity by 1.
            \end{itemize}
            \item ``Delete'' any saturated edges from the graph, to remove them from consideration when looking for additional paths $S\to T$.
        \end{enumerate}
        \item Repeat the process on the new \textit{augmented graph} until no more paths $S\to T$ exist. Then a max flow can be found by, for each edge $(v\to u)$ in the original network, assigning it the value $|(v\to u)|-|(u\to v)|$ from the augmented network.
    \end{enumerate}
\end{tcolorbox}

\subsubsection*{Proof of Correctness}
\begin{definition}
    Given an S-T network, we define a \textbf{cut} [partition] of the network to be a bipartition of the vertices of the network into two sets $S_1,S_2$ such that $S\in S_1$ and $T\in S_2$. \begin{itemize}
        \item Given a cut $(S,T)$, we define its \textbf{\textit{capacity}} $C(S,T)$ to be the sum of the capacities of all edges $(x\to y)$ such that $x\in S_1$, $y\in S_2$.
        
        $\rightarrow$ A \textbf{\textit{minimum cut}} is a cut with minimum capacity.\begin{itemize}
            \item Notably, $C(S,T)$ does not include edges $(y\to x)$ s.t. $x\in S_1$, $y\in S_2$.
        \end{itemize}
    \end{itemize}
\end{definition}

\begin{observation}
    Given a network with max flow $F$: for any cut $(S,T)$, we find that $|F|\leq C(S,T)$. In particular, $|F|$ is bounded above by the capacity of the minimum cut.\begin{itemize}
        \item \textit{Reasoning}: All flow in a max flow $F$ must pass from $S_1$ to $S_2$ at some point, but it is (by definition) not possible to send more than $C(S,T)$ flow between $S_1,S_2$.
    \end{itemize}
\end{observation}

\noindent\textit{}

\noindent\textit{Claim: The augmented greedy approach is optimal.}
\begin{proof}
Let $N$ be a network, $f$ a flow in $N$. We want to show that the following are equivalent: \begin{enumerate}
    \item $f$ is a max flow in $N$
    \item The augmented network $N_f$ ($N$ with flow $f$) contains no unsaturated augmented paths.
    \item $|f|=C(S,T)$ for some cut $(S,T)$ of $N$
\end{enumerate}

~\\
\noindent$1\to 2$: Assume $(1)$. Assume, for the sake of contradiction, that $N_f$ contains an unsaturated augmented path; then we could saturate that path and obtain a larger flow [contradiction].\\

\noindent$2\to 3$: Assume $(2)$. Remove (from $N_f$) all saturated edges. Then $S$ and $T$ will become disconnected in the resulting graph. \begin{itemize}
    \item \textit{Reasoning}: Since $N_f$ contains no unsaturated augmented paths, then every path $S\to T$ in $N_f$ must contain at least one saturated edge.
\end{itemize}

We can take the connected components containing $S$ and $T$, respectively, as a cut $(S,T)$ of $N$; then all edges between partitions are saturated in $N_f$, therefore $|f|=C(S,T)$.\\

\noindent$3\to 1$: Assume $(3)$. Since the capacity of any flow is bounded above by the capacity of any cut, then no flow $f'$ can have $f'>|f|=C(S,T)$; therefore $f$ is a max flow.
\end{proof}

\subsubsection*{Time Complexity}
Let $N(V,E)$ be a network with max flow $f$. Then Ford-Fulkerson will need to find at most $|f|$ augmenting paths, where finding an augmenting path via BFS/DFS is done in O($V+E$) and augmenting a graph is, worst-case, O($E$).\\

\noindent$\to$ \textbf{Runtime}: O($|f|\cdot E$) (\textit{pseudo-polynomial})

\pagebreak
\subsection{Extensions of Max Flow}
\subsubsection{Cell Tower Problem}
\begin{problem}[\probname{Cell Tower Problem}]
    Given a set of $m$ cell phones and $n$ cell towers (where each cell tower has a certain capacity), find a way to assign a maximum amount of cell phones to cell towers.\\

    \noindent\textbf{\textit{Assumptions}}:
    \begin{enumerate}
        \item Each cell tower has the same capacity $x$.
        \item Each cell phone can be assigned to any tower within a certain distance $R$.
    \end{enumerate}
\end{problem}

\noindent \textbf{Approach}: Can reframe as a network flow problem: \begin{enumerate}
    \item Represent cell phones, cell towers as individual nodes \begin{itemize}
        \item One phone-tower connection $\Rightarrow$ one unit of flow
        \item Can use edge capacities to enforce cell tower capacities
    \end{itemize}
\end{enumerate}

\vspace{8pt}
\noindent\ul{\textbf{Algorithm}}
\begin{enumerate}
    \item For each cell tower, create a corresponding node $y$ in the graph
    \item For each cell phone, create a corresponding node $x$ in the graph \begin{itemize}
        \item For each cell tower $y_i$ within a distance $R$ from $x$, create an edge $(x\to y_i)$ with capacity 1
    \end{itemize}
    \item Create a \textit{virtual source} $S$\begin{itemize}
        \item For each phone $a_i$,  create an edge $(a_i\to S)$ with capacity 1
    \end{itemize}
    \item Create a \textit{virtual sink} $T$ \begin{itemize}
        \item For each tower $b_j$, create an edge $(b_j\to T)$ with capacity $x$
    \end{itemize}
    \item Run Ford-Fulkerson from $S\to T$ on the graph.
\end{enumerate}

\pagebreak
\subsubsection{Bipartite Matching}
\begin{problem}[\probname{Bipartite Matching}]
    Given a graph partitioned into two sets $S_\text{left},S_\text{right}$, find the maximum number of possible matches $S_\text{left}\to S_\text{right}$ such that each node is matched with at most one other node.
\end{problem}

\noindent\ul{\textbf{Algorithm}}
\begin{enumerate}
    \item Create a \textit{virtual source} $S$\begin{itemize}
        \item For each node $x\in S_\text{left}$, create an edge $(S\to x)$ with capacity $1$
    \end{itemize}
    \item Create a \textit{virtual sink} $T$ \begin{itemize}
        \item For each node $y\in S_\text{right}$, create an edge $(y\to T)$ with capacity $1$
    \end{itemize}
    \item Run Ford-Fulkerson from $S\to T$ on the graph.
\end{enumerate}

\begin{center}
    \adjincludegraphics[trim={0 {0.4\height} 0 {0.2\height}},clip,scale=0.6]{Images/Notes/bipartite_flow.png}
\end{center}

\subsubsection{``Gadgets''}
If we want to create a restriction that at most $n$ flow passes through a given node $x$ in a network without modifying the capacities of adjacent edges, we can do so by splitting $x$ into two nodes $x_1,x_2$ such that:\begin{enumerate}
    \item All edges $(v\to x)$ become edges $(v\to x_1)$
    \item All edges $(x\to u)$ become edges $(x_2\to u)$
    \item $x_1,x_2$ are connected by an edge with capacity $n$
\end{enumerate}

\end{document}