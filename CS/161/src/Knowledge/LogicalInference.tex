\documentclass[12pt]{extarticle}
\usepackage[export]{adjustbox}
\usepackage{amsmath}
\usepackage{amssymb}
\usepackage{amsthm}
\usepackage{colortbl}
\usepackage{fancyhdr}
\usepackage[lmargin=0.9in,rmargin=0.9in,bmargin=1.25in]{geometry}
\usepackage{graphicx}
\usepackage{soul}
\usepackage{subfiles}
\usepackage[most]{tcolorbox}
\usepackage[explicit]{titlesec}
\usepackage{ulem}

\graphicspath{ {./../Images/Notes/} }

\title{CS161: Fundamentals of Artificial Intelligence}
\author{Stanley Wei}
\date{Prof. van den Broeck $\vert$ Winter 2024}

\titleformat{\subsection}{\large\bfseries}{\thesubsection}{1em}{#1\hrule\vspace*{-14pt}}

\titleformat{\subsubsection}
  {\normalfont\bfseries}{}{0pt}{\uline{#1}}

\theoremstyle{definition}
\newtheorem*{definition}{Definition}

\theoremstyle{remark}
\newtheorem*{example}{Ex}
\newtheorem*{note}{($\ast$) Note}

\newcommand{\pstart}[0]{\noindent}
\newcommand{\newp}[0]{~\\ \pstart}
\newcommand{\term}[1]{\noindent\textbf{\textit{#1}}}
\newcommand{\titleul}[1]{\noindent \textbf{\ul{#1}}}
\newcommand{\claim}[1]{\noindent Claim: \textit{#1}}
\newcommand{\resetcases}[0]{\setcounter{case}{0}}

\newcommand{\prob}[1]{\text{Pr}(#1)}
\newcommand{\cond}[2]{#1\,\vert\,#2}

\begin{document}
\term{Logical inference} is the process of checking if a knowledge base $KB$ entails a sentence $\alpha$. 

~\\ \pstart
Three strategies for logical inference: \begin{enumerate}
    \item \term{Enumerating models}: Checking if $M(KB)\subseteq M(\alpha)$
    \item \term{Deduction}: Using known logical patterns to try and \textit{deduce} $\alpha$ from the $KB$
    \item \term{Refutation}: Testing if assuming $KB\land\neg\alpha\implies UNSAT$
\end{enumerate} 

\subsection{Enumerating Models}
\term{Enumerating Models}: To check if $M(KB)\subseteq M(\alpha)$, build a truth table \& compare values of $KB$, $\alpha$ in each row [world]. \begin{itemize}
    \item If set of worlds satisfying $KB$ is subset of set of worlds satisfying $\alpha$ (every world with $\omega\models KB$ has $\omega\models\alpha$), then $KB\models\alpha$
    \item[($\ast$)] Drawback: Computationally expensive (exponential)
\end{itemize}

\subsection{Deduction}
\term{Deduction}: We can \textit{deduce} $\alpha$ from the $KB$ [$KB\vdash\alpha$] using \term{deduction rules}. \begin{itemize}
    \item Most efficient if the length of the proof is small
\end{itemize}

~\\ \pstart
\term{Deduction rules}: Logical statements that if certain patterns pattern$_1,\hdots,$ pattern$_n$ exist and hold, then a pattern$_{n+1}$ holds ($\frac{p_1,\hdots,p_n}{p_{n+1}}$): \begin{itemize}
    \item \textbf{Modus ponens}: If $\alpha\implies\beta$ and $\alpha$ holds, then $\beta$ holds [$\frac{\alpha,\alpha\implies\beta} {\beta}$]
    \item \textbf{Double negation}: If $\neg\neg\alpha$ holds, then $\alpha$ holds (and vice versa)
    \item \textbf{Or-introduction}: If $\alpha$ and $\beta$, then $\alpha\lor\beta$ holds
    \item \textbf{Bi-directional implication}: A bidrectional implication $\alpha\Longleftrightarrow\beta$ is equivalent to two unidirectional implications $\alpha\implies\beta$, $\beta\implies\alpha$
    \item \textbf{Contrapositive rule}: An implication $\alpha\implies\beta$ is logically equivalent to the reverse implication $\neg\beta\implies\neg\alpha$.
    \item \term{DeMorgan's Laws} relate to the negation of conjunctions/disjunctions: \begin{itemize}
        \item Negation of disjunction: $\neg(\alpha\lor\beta)\implies\neg\alpha\land\neg\beta$
        \item Negation of conjunction: $\neg(\alpha\land\beta)\implies\neg\alpha\lor\neg\beta$
    \end{itemize}
\end{itemize}

\newp
If we can use deduction rules to go from $KB$ to $\alpha$, have shown that $\alpha$ is entailed by the $KB$ \begin{itemize}
    \item[($\ast$)] Drawback: Have many rules, need to pick the right one [in humans: by intuition; machines: brute force]
\end{itemize}

\newp
For deduction to work, define various properties of deductive rules: \begin{itemize}
    \item \ul{Need} \term{soundness}: If $KB\vdash_R\alpha$, then $KB\models\alpha$. \begin{itemize}
        \item[($\ast$)] Ex. (Not sound): $A,A\implies B$ says $\neg B$
    \end{itemize}
    \item \ul{Want} \term{completeness}: If $KB\models\alpha$, then $KB\vdash_R\alpha$  \begin{itemize}
        \item[($\ast$)] Ex. (Sound, not complete): modus ponens $A,A\implies B$ says $B$ [ex: if $KB=A\cap B$ and $\alpha=B$, modus ponens does not give ]
        \item[($\ast$)] Ex. (Complete, not sound): $\frac{\;\;\;~}{\alpha}$
    \end{itemize}
\end{itemize}


\subsection{Refutation}
Say a rule is \term{refutation-complete} if: $KB\models False$ $\implies$ $KB\vdash False$. \begin{itemize}
    \item Weaker version of completeness
    \item If rule sound and refutation-complete, the $KB\models$ False iff $KB\vdash_R$ False [via entailment-to-SAT] \begin{itemize}
        \item Can check if $KB\vdash_R$ to check if $K\models$ True
    \end{itemize}
\end{itemize}

\newp
$\Rightarrow$ \term{Proof by refutation}: To determine if $KB\models\alpha$, test if $KB\land\neg\alpha\implies$ False/UNSAT.\begin{itemize}
    \item How to test: from $KB$, generate $KB'=KB\land\neg\alpha$; convert to CNF, test if valid
\end{itemize}

\newp
New rule: \term{resolution} - $\frac{\alpha\lor e,\,\neg e\lor\gamma}{\alpha\lor\gamma}$ \begin{itemize}
    \item Sound, not complete: $KB= A\lor B,\alpha=A\lor B\lor C$
    \item Is refutation-complete if $KB$ expressed in CNF \begin{itemize}
        \item Resolution works on clauses/disjunctions
    \end{itemize}
\end{itemize}

\newp
\term{Resolution algorithm}: Take $KB\land\neg\alpha$, keep applying Resolution, see if $KB\land\neg\alpha\vdash$ False (if so, implies $KB\land\neg\alpha\models$ False, therefore $KB\models\alpha$) \begin{itemize}
    \item If $KB$ runs out of rules (we cannot prove False; $KB\land\neg\alpha\not\vdash$ False), then per refutation-completeness, $KB\land\neg\alpha\not\models$ False; then $KB\not\models\alpha$
\end{itemize}

\subsection{DPLL}
Alternative approach: \term{SAT solving/DPLL Search} uses the constraint satisfaction algorithm (as seen previously) to perform refutation \begin{enumerate}
    \item Take individual literals/variables $A,B,C,\hdots$ as variables in CSP (possible values: True, False)
    \item Take the set of CNF clauses as a set of CSP constraints
    \item Use CSP to check if there is a set of assignments that satisfies $KB\land\neg\alpha$ [refutation] 

    \item[($\ast$)] Basic DPLL algorithm (SAT($F$)) - Given a set of clauses $F$: \begin{itemize}
            \item If $F=$ \{False\}, return 0
            \item If $F=$ \{True\}, return 1
            \item Else, return $SAT(F_{x=0})\lor SAT(F_{x=1})$
        \end{itemize}
\end{enumerate}

\newp
Want to do constraint propagation (similar to forward checking/arc consistency)

\vspace{4pt}\pstart
$\Rightarrow$ \term{Unit resolution/unit propagation}: If $F$ has a sentence $\alpha$ [single variable] and a sentence $\neg\alpha\lor\beta$, can infer the sentence $\beta$ \begin{itemize}
        \item Either removes a clause, or removes literal from a clause [simplifies clause]
        \item[$\ast$] Specialized form of arc consistency/forward checking
    \end{itemize}

\newp
($\ast$) \textit{DPLL Optimizations}: \begin{itemize}
    \item If a conflict clause (a simple clause entailed by the original $KB$) is found, can stop early and restart the search, incorporating that clause into the $KB$
    \item \textit{Backjumping}: If a failure was caused by a variable assignment that was made 10 choices ago, backtrack 10 levels in search tree
    \item If the CNF is found to have separate components, can solve each component separately
    \item \textit{Pure literal}: If a variable $X$ only appears as positive literal (i.e. only as $X$, never as $\neg X$), can immediately set $X$ to be True \begin{itemize}
        \item If there is a solution, there exists a solution with $X$ = True; setting $X$ to true from the start makes problem easier (eliminates clauses with $X$)
    \end{itemize}
    \item Can use heuristics to determine variable, value assignments
    \item Can convert the CSP into a graph to find symmetries \begin{itemize}
        \item Helps to avoid evaluating symmetric solutions
    \end{itemize}
    \item For a very long clause, can designate two variables inside of it as ``special'' and only update the clause value when one of two specials is assigned \begin{itemize}
        \item Avoids computational cost of re-evaluating clause for every assignment
    \end{itemize}
\end{itemize}

\newp
Can also use local search for faster SAT-solving (rather than full backtracking search) \begin{itemize}
    \item \textit{Issue}: If local search fails to find satisfying assignment, then either none exists, or one simply hasn't been found \begin{itemize}
        \item Can only prove $KB\not\models\alpha$, if a SAT assignment is found; could never prove $KB\models\alpha$ without somehow knowing that all states have been seen
    \end{itemize}
\end{itemize}

\end{document}
